
\begin{abstract}
Las técnicas de regresión son versátiles en su aplicación a la investigación médica, ya que permiten medir asociaciones, predecir resultados y controlar efectos de variables de confusión. Como una de estas técnicas, la regresión logística representa una forma eficiente y poderosa de analizar el efecto de un grupo de variables independientes sobre un resultado binario, cuantificando la contribución única de cada variable. Usando componentes de la regresión lineal reflejados en la escala logit, la regresió...

Consideraciones importantes al aplicar regresión logística incluyen: la selección adecuada de variables independientes, el cumplimiento de los supuestos necesarios, y la elección de una estrategia adecuada de construcción del modelo. La selección de variables debe guiarse por teorías aceptadas, investigaciones empíricas previas, consideraciones clínicas y análisis estadísticos univariados, incluyendo el reconocimiento de posibles variables de confusión.

Entre los supuestos básicos que deben cumplirse se encuentran: independencia de errores, linealidad en la escala logit para variables continuas, ausencia de multicolinealidad y falta de valores atípicos con fuerte influencia. Además, debe asegurarse un número adecuado de eventos por variable para evitar el sobreajuste, recomendándose comúnmente entre 10 y 20 eventos por covariable.

En cuanto a las estrategias de modelado, existen tres tipos generales: directa/estándar, secuencial/jerárquica y por pasos/estadística, cada una con distinto énfasis y propósito. Antes de extraer conclusiones definitivas, se recomienda cuantificar formalmente la validez interna del modelo (es decir, su replicabilidad en el mismo conjunto de datos) y su validez externa (generalización a otros conjuntos de datos).

El ajuste general del modelo al conjunto de datos se evalúa mediante diversas medidas de bondad de ajuste, siendo preferible un menor error entre los valores observados y los predichos. También se recomienda el uso de estadísticas diagnósticas para valorar la adecuación del modelo. Finalmente, los resultados para las variables independientes se reportan típicamente como razones de momios (odds ratios, ORs) con intervalos de confianza al 95\%.
\end{abstract}
