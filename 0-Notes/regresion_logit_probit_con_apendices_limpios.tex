% Options for packages loaded elsewhere
\PassOptionsToPackage{unicode}{hyperref}
\PassOptionsToPackage{hyphens}{url}
%
\documentclass[
]{article}
\usepackage{amsmath,amssymb}
\usepackage{iftex}
\ifPDFTeX
  \usepackage[T1]{fontenc}
  \usepackage[utf8]{inputenc}
  \usepackage{textcomp} % provide euro and other symbols
\else % if luatex or xetex
  \usepackage{unicode-math} % this also loads fontspec
  \defaultfontfeatures{Scale=MatchLowercase}
  \defaultfontfeatures[\rmfamily]{Ligatures=TeX,Scale=1}
\fi
\usepackage{lmodern}
\ifPDFTeX\else
  % xetex/luatex font selection
\fi
% Use upquote if available, for straight quotes in verbatim environments
\IfFileExists{upquote.sty}{\usepackage{upquote}}{}
\IfFileExists{microtype.sty}{% use microtype if available
  \usepackage[]{microtype}
  \UseMicrotypeSet[protrusion]{basicmath} % disable protrusion for tt fonts
}{}
\makeatletter
\@ifundefined{KOMAClassName}{% if non-KOMA class
  \IfFileExists{parskip.sty}{%
    \usepackage{parskip}
  }{% else
    \setlength{\parindent}{0pt}
    \setlength{\parskip}{6pt plus 2pt minus 1pt}}
}{% if KOMA class
  \KOMAoptions{parskip=half}}
\makeatother
\usepackage{xcolor}
\usepackage[margin=1in]{geometry}
\usepackage{color}
\usepackage{fancyvrb}
\newcommand{\VerbBar}{|}
\newcommand{\VERB}{\Verb[commandchars=\\\{\}]}
\DefineVerbatimEnvironment{Highlighting}{Verbatim}{commandchars=\\\{\}}
% Add ',fontsize=\small' for more characters per line
\usepackage{framed}
\definecolor{shadecolor}{RGB}{248,248,248}
\newenvironment{Shaded}{\begin{snugshade}}{\end{snugshade}}
\newcommand{\AlertTok}[1]{\textcolor[rgb]{0.94,0.16,0.16}{#1}}
\newcommand{\AnnotationTok}[1]{\textcolor[rgb]{0.56,0.35,0.01}{\textbf{\textit{#1}}}}
\newcommand{\AttributeTok}[1]{\textcolor[rgb]{0.13,0.29,0.53}{#1}}
\newcommand{\BaseNTok}[1]{\textcolor[rgb]{0.00,0.00,0.81}{#1}}
\newcommand{\BuiltInTok}[1]{#1}
\newcommand{\CharTok}[1]{\textcolor[rgb]{0.31,0.60,0.02}{#1}}
\newcommand{\CommentTok}[1]{\textcolor[rgb]{0.56,0.35,0.01}{\textit{#1}}}
\newcommand{\CommentVarTok}[1]{\textcolor[rgb]{0.56,0.35,0.01}{\textbf{\textit{#1}}}}
\newcommand{\ConstantTok}[1]{\textcolor[rgb]{0.56,0.35,0.01}{#1}}
\newcommand{\ControlFlowTok}[1]{\textcolor[rgb]{0.13,0.29,0.53}{\textbf{#1}}}
\newcommand{\DataTypeTok}[1]{\textcolor[rgb]{0.13,0.29,0.53}{#1}}
\newcommand{\DecValTok}[1]{\textcolor[rgb]{0.00,0.00,0.81}{#1}}
\newcommand{\DocumentationTok}[1]{\textcolor[rgb]{0.56,0.35,0.01}{\textbf{\textit{#1}}}}
\newcommand{\ErrorTok}[1]{\textcolor[rgb]{0.64,0.00,0.00}{\textbf{#1}}}
\newcommand{\ExtensionTok}[1]{#1}
\newcommand{\FloatTok}[1]{\textcolor[rgb]{0.00,0.00,0.81}{#1}}
\newcommand{\FunctionTok}[1]{\textcolor[rgb]{0.13,0.29,0.53}{\textbf{#1}}}
\newcommand{\ImportTok}[1]{#1}
\newcommand{\InformationTok}[1]{\textcolor[rgb]{0.56,0.35,0.01}{\textbf{\textit{#1}}}}
\newcommand{\KeywordTok}[1]{\textcolor[rgb]{0.13,0.29,0.53}{\textbf{#1}}}
\newcommand{\NormalTok}[1]{#1}
\newcommand{\OperatorTok}[1]{\textcolor[rgb]{0.81,0.36,0.00}{\textbf{#1}}}
\newcommand{\OtherTok}[1]{\textcolor[rgb]{0.56,0.35,0.01}{#1}}
\newcommand{\PreprocessorTok}[1]{\textcolor[rgb]{0.56,0.35,0.01}{\textit{#1}}}
\newcommand{\RegionMarkerTok}[1]{#1}
\newcommand{\SpecialCharTok}[1]{\textcolor[rgb]{0.81,0.36,0.00}{\textbf{#1}}}
\newcommand{\SpecialStringTok}[1]{\textcolor[rgb]{0.31,0.60,0.02}{#1}}
\newcommand{\StringTok}[1]{\textcolor[rgb]{0.31,0.60,0.02}{#1}}
\newcommand{\VariableTok}[1]{\textcolor[rgb]{0.00,0.00,0.00}{#1}}
\newcommand{\VerbatimStringTok}[1]{\textcolor[rgb]{0.31,0.60,0.02}{#1}}
\newcommand{\WarningTok}[1]{\textcolor[rgb]{0.56,0.35,0.01}{\textbf{\textit{#1}}}}
\usepackage{graphicx}
\makeatletter
\def\maxwidth{\ifdim\Gin@nat@width>\linewidth\linewidth\else\Gin@nat@width\fi}
\def\maxheight{\ifdim\Gin@nat@height>\textheight\textheight\else\Gin@nat@height\fi}
\makeatother
% Scale images if necessary, so that they will not overflow the page
% margins by default, and it is still possible to overwrite the defaults
% using explicit options in \includegraphics[width, height, ...]{}
\setkeys{Gin}{width=\maxwidth,height=\maxheight,keepaspectratio}
% Set default figure placement to htbp
\makeatletter
\def\fps@figure{htbp}
\makeatother
\setlength{\emergencystretch}{3em} % prevent overfull lines
\providecommand{\tightlist}{%
  \setlength{\itemsep}{0pt}\setlength{\parskip}{0pt}}
\setcounter{secnumdepth}{5}
\ifLuaTeX
  \usepackage{selnolig}  % disable illegal ligatures
\fi
\usepackage{bookmark}
\IfFileExists{xurl.sty}{\usepackage{xurl}}{} % add URL line breaks if available
\urlstyle{same}
\hypersetup{
  pdftitle={Análisis de Regresión Logística, Probit y Bootstrap},
  pdfauthor={Carlos},
  hidelinks,
  pdfcreator={LaTeX via pandoc}}

\title{Análisis de Regresión Logística, Probit y Bootstrap}
\author{Carlos}
\date{2025-04-03}

\begin{document}
\maketitle

{
\setcounter{tocdepth}{2}
\tableofcontents
}
\subsection{Carga de Datos}\label{carga-de-datos}

\begin{Shaded}
\begin{Highlighting}[]
\NormalTok{auto\_time }\OtherTok{\textless{}{-}} \FunctionTok{c}\NormalTok{(}\FloatTok{52.90}\NormalTok{, }\FloatTok{4.10}\NormalTok{, }\FloatTok{4.10}\NormalTok{, }\FloatTok{56.20}\NormalTok{, }\FloatTok{51.80}\NormalTok{, }\FloatTok{0.20}\NormalTok{, }\FloatTok{27.60}\NormalTok{, }\FloatTok{89.90}\NormalTok{, }\FloatTok{41.50}\NormalTok{,}
               \FloatTok{95.00}\NormalTok{, }\FloatTok{99.10}\NormalTok{, }\FloatTok{18.50}\NormalTok{, }\FloatTok{82.00}\NormalTok{, }\FloatTok{8.60}\NormalTok{, }\FloatTok{22.50}\NormalTok{, }\FloatTok{51.40}\NormalTok{, }\FloatTok{81.00}\NormalTok{, }\FloatTok{51.00}\NormalTok{,}
               \FloatTok{62.20}\NormalTok{, }\FloatTok{95.10}\NormalTok{, }\FloatTok{41.60}\NormalTok{)}
\NormalTok{bus\_time }\OtherTok{\textless{}{-}} \FunctionTok{c}\NormalTok{(}\FloatTok{4.4}\NormalTok{, }\FloatTok{28.5}\NormalTok{, }\FloatTok{86.9}\NormalTok{, }\FloatTok{31.6}\NormalTok{, }\FloatTok{20.2}\NormalTok{, }\FloatTok{91.2}\NormalTok{, }\FloatTok{79.7}\NormalTok{, }\FloatTok{2.2}\NormalTok{, }\FloatTok{24.5}\NormalTok{,}
              \FloatTok{43.5}\NormalTok{, }\FloatTok{8.4}\NormalTok{, }\DecValTok{84}\NormalTok{, }\DecValTok{38}\NormalTok{, }\FloatTok{1.6}\NormalTok{, }\FloatTok{74.1}\NormalTok{, }\FloatTok{83.8}\NormalTok{, }\FloatTok{19.2}\NormalTok{, }\DecValTok{85}\NormalTok{, }\FloatTok{90.1}\NormalTok{, }\FloatTok{22.2}\NormalTok{, }\FloatTok{91.5}\NormalTok{)}
\NormalTok{y }\OtherTok{\textless{}{-}} \FunctionTok{c}\NormalTok{(}\DecValTok{0}\NormalTok{, }\DecValTok{0}\NormalTok{, }\DecValTok{1}\NormalTok{, }\DecValTok{0}\NormalTok{, }\DecValTok{0}\NormalTok{, }\DecValTok{1}\NormalTok{, }\DecValTok{1}\NormalTok{, }\DecValTok{0}\NormalTok{, }\DecValTok{0}\NormalTok{,}
       \DecValTok{0}\NormalTok{, }\DecValTok{0}\NormalTok{, }\DecValTok{1}\NormalTok{, }\DecValTok{1}\NormalTok{, }\DecValTok{0}\NormalTok{, }\DecValTok{1}\NormalTok{, }\DecValTok{1}\NormalTok{, }\DecValTok{0}\NormalTok{, }\DecValTok{1}\NormalTok{, }\DecValTok{1}\NormalTok{, }\DecValTok{0}\NormalTok{, }\DecValTok{1}\NormalTok{)}
\NormalTok{x }\OtherTok{\textless{}{-}}\NormalTok{ auto\_time }\SpecialCharTok{{-}}\NormalTok{ bus\_time}
\end{Highlighting}
\end{Shaded}

\subsection{Modelo de Regresión
Logística}\label{modelo-de-regresiuxf3n-loguxedstica}

\begin{Shaded}
\begin{Highlighting}[]
\NormalTok{modelo }\OtherTok{\textless{}{-}} \FunctionTok{glm}\NormalTok{(y }\SpecialCharTok{\textasciitilde{}}\NormalTok{ x, }\AttributeTok{family =}\NormalTok{ binomial)}
\FunctionTok{summary}\NormalTok{(modelo)}
\end{Highlighting}
\end{Shaded}

\begin{verbatim}
## 
## Call:
## glm(formula = y ~ x, family = binomial)
## 
## Coefficients:
##             Estimate Std. Error z value Pr(>|z|)  
## (Intercept) -0.23758    0.75048  -0.317   0.7516  
## x           -0.05311    0.02064  -2.573   0.0101 *
## ---
## Signif. codes:  0 '***' 0.001 '**' 0.01 '*' 0.05 '.' 0.1 ' ' 1
## 
## (Dispersion parameter for binomial family taken to be 1)
## 
##     Null deviance: 29.065  on 20  degrees of freedom
## Residual deviance: 12.332  on 19  degrees of freedom
## AIC: 16.332
## 
## Number of Fisher Scoring iterations: 6
\end{verbatim}

\begin{Shaded}
\begin{Highlighting}[]
\FunctionTok{plot}\NormalTok{(x, y, }\AttributeTok{pch =} \DecValTok{19}\NormalTok{, }\AttributeTok{col =} \FunctionTok{ifelse}\NormalTok{(y }\SpecialCharTok{==} \DecValTok{1}\NormalTok{, }\StringTok{"blue"}\NormalTok{, }\StringTok{"red"}\NormalTok{),}
     \AttributeTok{main =} \StringTok{"Logistic Regression Fit"}\NormalTok{, }\AttributeTok{xlab =} \StringTok{"x = Auto Time {-} Bus Time"}\NormalTok{, }\AttributeTok{ylab =} \StringTok{"P(car)"}\NormalTok{)}
\FunctionTok{curve}\NormalTok{(}\FunctionTok{predict}\NormalTok{(modelo, }\FunctionTok{data.frame}\NormalTok{(}\AttributeTok{x =} \FunctionTok{sort}\NormalTok{(x)), }\AttributeTok{type =} \StringTok{"response"}\NormalTok{),}
      \AttributeTok{add =} \ConstantTok{TRUE}\NormalTok{, }\AttributeTok{col =} \StringTok{"darkgreen"}\NormalTok{, }\AttributeTok{lwd =} \DecValTok{2}\NormalTok{)}
\FunctionTok{legend}\NormalTok{(}\StringTok{"bottomright"}\NormalTok{, }\AttributeTok{legend =} \FunctionTok{c}\NormalTok{(}\StringTok{"Observado 1"}\NormalTok{, }\StringTok{"Observado 0"}\NormalTok{, }\StringTok{"Curva logística"}\NormalTok{),}
       \AttributeTok{col =} \FunctionTok{c}\NormalTok{(}\StringTok{"blue"}\NormalTok{, }\StringTok{"red"}\NormalTok{, }\StringTok{"darkgreen"}\NormalTok{), }\AttributeTok{pch =} \FunctionTok{c}\NormalTok{(}\DecValTok{19}\NormalTok{, }\DecValTok{19}\NormalTok{, }\ConstantTok{NA}\NormalTok{), }\AttributeTok{lty =} \FunctionTok{c}\NormalTok{(}\ConstantTok{NA}\NormalTok{, }\ConstantTok{NA}\NormalTok{, }\DecValTok{1}\NormalTok{))}
\end{Highlighting}
\end{Shaded}

\includegraphics{regresion_logit_probit_con_apendices_limpios_files/figure-latex/unnamed-chunk-2-1.pdf}

\subsection{Ajuste por Mínimos Cuadrados No Lineales
(NLS)}\label{ajuste-por-muxednimos-cuadrados-no-lineales-nls}

\begin{Shaded}
\begin{Highlighting}[]
\NormalTok{b0\_init }\OtherTok{\textless{}{-}} \FunctionTok{coef}\NormalTok{(modelo)[}\DecValTok{1}\NormalTok{]}
\NormalTok{b1\_init }\OtherTok{\textless{}{-}} \FunctionTok{coef}\NormalTok{(modelo)[}\DecValTok{2}\NormalTok{]}
\NormalTok{modelo\_nls }\OtherTok{\textless{}{-}} \FunctionTok{nls}\NormalTok{(}
\NormalTok{  y }\SpecialCharTok{\textasciitilde{}} \DecValTok{1} \SpecialCharTok{/}\NormalTok{ (}\DecValTok{1} \SpecialCharTok{+} \FunctionTok{exp}\NormalTok{(}\SpecialCharTok{{-}}\NormalTok{(b0 }\SpecialCharTok{+}\NormalTok{ b1 }\SpecialCharTok{*}\NormalTok{ x))),}
  \AttributeTok{start =} \FunctionTok{list}\NormalTok{(}\AttributeTok{b0 =}\NormalTok{ b0\_init , }\AttributeTok{b1 =}\NormalTok{ b1\_init)}
\NormalTok{)}
\FunctionTok{summary}\NormalTok{(modelo\_nls)}
\end{Highlighting}
\end{Shaded}

\begin{verbatim}
## 
## Formula: y ~ 1/(1 + exp(-(b0 + b1 * x)))
## 
## Parameters:
##      Estimate Std. Error t value Pr(>|t|)
## b0   -188.344 744010.283       0        1
## b1     -7.197  27886.344       0        1
## 
## Residual standard error: 0.2294 on 19 degrees of freedom
## 
## Number of iterations to convergence: 16 
## Achieved convergence tolerance: 4.887e-06
\end{verbatim}

\begin{Shaded}
\begin{Highlighting}[]
\FunctionTok{plot}\NormalTok{(x, y, }\AttributeTok{pch =} \DecValTok{19}\NormalTok{, }\AttributeTok{col =} \FunctionTok{ifelse}\NormalTok{(y }\SpecialCharTok{==} \DecValTok{1}\NormalTok{, }\StringTok{"blue"}\NormalTok{, }\StringTok{"red"}\NormalTok{),}
     \AttributeTok{main =} \StringTok{"Comparación de Métodos de Ajuste"}\NormalTok{, }\AttributeTok{xlab =} \StringTok{"x = Auto Time {-} Bus Time"}\NormalTok{, }\AttributeTok{ylab =} \StringTok{"P(car)"}\NormalTok{)}
\NormalTok{x\_sorted }\OtherTok{\textless{}{-}} \FunctionTok{sort}\NormalTok{(x)}
\FunctionTok{lines}\NormalTok{(x\_sorted, }\FunctionTok{predict}\NormalTok{(modelo, }\AttributeTok{type =} \StringTok{"response"}\NormalTok{)[}\FunctionTok{order}\NormalTok{(x)], }\AttributeTok{col =} \StringTok{"darkgreen"}\NormalTok{, }\AttributeTok{lwd =} \DecValTok{2}\NormalTok{)}

\NormalTok{b0\_nls }\OtherTok{\textless{}{-}} \FunctionTok{coef}\NormalTok{(modelo\_nls)[}\StringTok{"b0"}\NormalTok{]}
\NormalTok{b1\_nls }\OtherTok{\textless{}{-}} \FunctionTok{coef}\NormalTok{(modelo\_nls)[}\StringTok{"b1"}\NormalTok{]}
\NormalTok{p\_nls }\OtherTok{\textless{}{-}} \DecValTok{1} \SpecialCharTok{/}\NormalTok{ (}\DecValTok{1} \SpecialCharTok{+} \FunctionTok{exp}\NormalTok{(}\SpecialCharTok{{-}}\NormalTok{(b0\_nls }\SpecialCharTok{+}\NormalTok{ b1\_nls }\SpecialCharTok{*}\NormalTok{ x\_sorted)))}
\FunctionTok{lines}\NormalTok{(x\_sorted, p\_nls, }\AttributeTok{col =} \StringTok{"purple"}\NormalTok{, }\AttributeTok{lwd =} \DecValTok{2}\NormalTok{, }\AttributeTok{lty =} \DecValTok{2}\NormalTok{)}

\FunctionTok{legend}\NormalTok{(}\StringTok{"bottomleft"}\NormalTok{, }\AttributeTok{legend =} \FunctionTok{c}\NormalTok{(}\StringTok{"GLM (Verosimilitud)"}\NormalTok{, }\StringTok{"NLS (Mínimos Cuadrados)"}\NormalTok{),}
       \AttributeTok{col =} \FunctionTok{c}\NormalTok{(}\StringTok{"darkgreen"}\NormalTok{, }\StringTok{"purple"}\NormalTok{), }\AttributeTok{lty =} \FunctionTok{c}\NormalTok{(}\DecValTok{1}\NormalTok{, }\DecValTok{2}\NormalTok{), }\AttributeTok{lwd =} \DecValTok{2}\NormalTok{, }\AttributeTok{cex =} \FloatTok{0.5}\NormalTok{)}
\end{Highlighting}
\end{Shaded}

\includegraphics{regresion_logit_probit_con_apendices_limpios_files/figure-latex/unnamed-chunk-3-1.pdf}

\subsection{Modelo Probit}\label{modelo-probit}

\begin{Shaded}
\begin{Highlighting}[]
\NormalTok{modelo\_probit }\OtherTok{\textless{}{-}} \FunctionTok{glm}\NormalTok{(y }\SpecialCharTok{\textasciitilde{}}\NormalTok{ x, }\AttributeTok{family =} \FunctionTok{binomial}\NormalTok{(}\AttributeTok{link =} \StringTok{"probit"}\NormalTok{))}
\FunctionTok{summary}\NormalTok{(modelo\_probit)}
\end{Highlighting}
\end{Shaded}

\begin{verbatim}
## 
## Call:
## glm(formula = y ~ x, family = binomial(link = "probit"))
## 
## Coefficients:
##             Estimate Std. Error z value Pr(>|z|)   
## (Intercept) -0.06443    0.40068  -0.161  0.87224   
## x           -0.03000    0.01029  -2.915  0.00355 **
## ---
## Signif. codes:  0 '***' 0.001 '**' 0.01 '*' 0.05 '.' 0.1 ' ' 1
## 
## (Dispersion parameter for binomial family taken to be 1)
## 
##     Null deviance: 29.065  on 20  degrees of freedom
## Residual deviance: 12.330  on 19  degrees of freedom
## AIC: 16.33
## 
## Number of Fisher Scoring iterations: 7
\end{verbatim}

\begin{Shaded}
\begin{Highlighting}[]
\FunctionTok{plot}\NormalTok{(x, y, }\AttributeTok{pch =} \DecValTok{19}\NormalTok{, }\AttributeTok{col =} \FunctionTok{ifelse}\NormalTok{(y }\SpecialCharTok{==} \DecValTok{1}\NormalTok{, }\StringTok{"blue"}\NormalTok{, }\StringTok{"red"}\NormalTok{),}
     \AttributeTok{main =} \StringTok{"Ajuste de Regresión Probit"}\NormalTok{, }\AttributeTok{xlab =} \StringTok{"x = Auto Time {-} Bus Time"}\NormalTok{, }\AttributeTok{ylab =} \StringTok{"P(car)"}\NormalTok{)}

\NormalTok{x\_seq }\OtherTok{\textless{}{-}} \FunctionTok{seq}\NormalTok{(}\FunctionTok{min}\NormalTok{(x), }\FunctionTok{max}\NormalTok{(x), }\AttributeTok{length.out =} \DecValTok{100}\NormalTok{)}
\FunctionTok{lines}\NormalTok{(x\_seq, }\FunctionTok{predict}\NormalTok{(modelo\_probit, }\AttributeTok{newdata =} \FunctionTok{data.frame}\NormalTok{(}\AttributeTok{x =}\NormalTok{ x\_seq), }\AttributeTok{type =} \StringTok{"response"}\NormalTok{),}
      \AttributeTok{col =} \StringTok{"orange"}\NormalTok{, }\AttributeTok{lwd =} \DecValTok{2}\NormalTok{)}

\FunctionTok{legend}\NormalTok{(}\StringTok{"bottomleft"}\NormalTok{, }\AttributeTok{legend =} \FunctionTok{c}\NormalTok{(}\StringTok{"Observado 1"}\NormalTok{, }\StringTok{"Observado 0"}\NormalTok{, }\StringTok{"Curva Probit"}\NormalTok{),}
       \AttributeTok{col =} \FunctionTok{c}\NormalTok{(}\StringTok{"blue"}\NormalTok{, }\StringTok{"red"}\NormalTok{, }\StringTok{"orange"}\NormalTok{), }\AttributeTok{pch =} \FunctionTok{c}\NormalTok{(}\DecValTok{19}\NormalTok{, }\DecValTok{19}\NormalTok{, }\ConstantTok{NA}\NormalTok{), }\AttributeTok{lty =} \FunctionTok{c}\NormalTok{(}\ConstantTok{NA}\NormalTok{, }\ConstantTok{NA}\NormalTok{, }\DecValTok{1}\NormalTok{), }\AttributeTok{lwd =} \DecValTok{2}\NormalTok{, }\AttributeTok{cex =} \FloatTok{0.8}\NormalTok{)}
\end{Highlighting}
\end{Shaded}

\includegraphics{regresion_logit_probit_con_apendices_limpios_files/figure-latex/unnamed-chunk-4-1.pdf}

\subsection{Comparación: Logit vs
Probit}\label{comparaciuxf3n-logit-vs-probit}

\begin{Shaded}
\begin{Highlighting}[]
\NormalTok{modelo\_logit  }\OtherTok{\textless{}{-}} \FunctionTok{glm}\NormalTok{(y }\SpecialCharTok{\textasciitilde{}}\NormalTok{ x, }\AttributeTok{family =} \FunctionTok{binomial}\NormalTok{(}\AttributeTok{link =} \StringTok{"logit"}\NormalTok{))}

\FunctionTok{plot}\NormalTok{(x, y, }\AttributeTok{pch =} \DecValTok{19}\NormalTok{, }\AttributeTok{col =} \FunctionTok{ifelse}\NormalTok{(y }\SpecialCharTok{==} \DecValTok{1}\NormalTok{, }\StringTok{"blue"}\NormalTok{, }\StringTok{"red"}\NormalTok{),}
     \AttributeTok{main =} \StringTok{"Comparación: Logit vs Probit"}\NormalTok{, }\AttributeTok{xlab =} \StringTok{"x = Auto Time {-} Bus Time"}\NormalTok{, }\AttributeTok{ylab =} \StringTok{"P(car)"}\NormalTok{, }\AttributeTok{ylim =} \FunctionTok{c}\NormalTok{(}\DecValTok{0}\NormalTok{, }\DecValTok{1}\NormalTok{))}

\FunctionTok{lines}\NormalTok{(x\_seq, }\FunctionTok{predict}\NormalTok{(modelo\_logit, }\AttributeTok{newdata =} \FunctionTok{data.frame}\NormalTok{(}\AttributeTok{x =}\NormalTok{ x\_seq), }\AttributeTok{type =} \StringTok{"response"}\NormalTok{),}
      \AttributeTok{col =} \StringTok{"darkgreen"}\NormalTok{, }\AttributeTok{lwd =} \DecValTok{2}\NormalTok{)}
\FunctionTok{lines}\NormalTok{(x\_seq, }\FunctionTok{predict}\NormalTok{(modelo\_probit, }\AttributeTok{newdata =} \FunctionTok{data.frame}\NormalTok{(}\AttributeTok{x =}\NormalTok{ x\_seq), }\AttributeTok{type =} \StringTok{"response"}\NormalTok{),}
      \AttributeTok{col =} \StringTok{"orange"}\NormalTok{, }\AttributeTok{lwd =} \DecValTok{2}\NormalTok{, }\AttributeTok{lty =} \DecValTok{2}\NormalTok{)}

\FunctionTok{legend}\NormalTok{(}\StringTok{"bottomleft"}\NormalTok{, }\AttributeTok{legend =} \FunctionTok{c}\NormalTok{(}\StringTok{"Observado 1"}\NormalTok{, }\StringTok{"Observado 0"}\NormalTok{, }\StringTok{"Logit"}\NormalTok{, }\StringTok{"Probit"}\NormalTok{),}
       \AttributeTok{col =} \FunctionTok{c}\NormalTok{(}\StringTok{"blue"}\NormalTok{, }\StringTok{"red"}\NormalTok{, }\StringTok{"darkgreen"}\NormalTok{, }\StringTok{"orange"}\NormalTok{),}
       \AttributeTok{pch =} \FunctionTok{c}\NormalTok{(}\DecValTok{19}\NormalTok{, }\DecValTok{19}\NormalTok{, }\ConstantTok{NA}\NormalTok{, }\ConstantTok{NA}\NormalTok{), }\AttributeTok{lty =} \FunctionTok{c}\NormalTok{(}\ConstantTok{NA}\NormalTok{, }\ConstantTok{NA}\NormalTok{, }\DecValTok{1}\NormalTok{, }\DecValTok{2}\NormalTok{), }\AttributeTok{lwd =} \DecValTok{2}\NormalTok{, }\AttributeTok{cex =} \FloatTok{0.5}\NormalTok{)}
\end{Highlighting}
\end{Shaded}

\includegraphics{regresion_logit_probit_con_apendices_limpios_files/figure-latex/unnamed-chunk-5-1.pdf}

\subsection{Bootstrap + Modelos}\label{bootstrap-modelos}

\begin{Shaded}
\begin{Highlighting}[]
\NormalTok{datos }\OtherTok{\textless{}{-}} \FunctionTok{data.frame}\NormalTok{(auto\_time, bus\_time, }\AttributeTok{x =}\NormalTok{ auto\_time }\SpecialCharTok{{-}}\NormalTok{ bus\_time, y)}
\FunctionTok{set.seed}\NormalTok{(}\DecValTok{123}\NormalTok{)}
\NormalTok{bootstrap\_sample }\OtherTok{\textless{}{-}}\NormalTok{ datos[}\FunctionTok{sample}\NormalTok{(}\DecValTok{1}\SpecialCharTok{:}\FunctionTok{nrow}\NormalTok{(datos), }\AttributeTok{size =} \DecValTok{100}\NormalTok{, }\AttributeTok{replace =} \ConstantTok{TRUE}\NormalTok{), ]}
\NormalTok{x\_btsp }\OtherTok{\textless{}{-}}\NormalTok{ bootstrap\_sample}\SpecialCharTok{$}\NormalTok{x}
\NormalTok{y\_btsp }\OtherTok{\textless{}{-}}\NormalTok{ bootstrap\_sample}\SpecialCharTok{$}\NormalTok{y}

\NormalTok{modelo\_btstrap }\OtherTok{\textless{}{-}} \FunctionTok{glm}\NormalTok{(y\_btsp }\SpecialCharTok{\textasciitilde{}}\NormalTok{ x\_btsp, }\AttributeTok{family =}\NormalTok{ binomial)}
\FunctionTok{summary}\NormalTok{(modelo\_btstrap)}
\end{Highlighting}
\end{Shaded}

\begin{verbatim}
## 
## Call:
## glm(formula = y_btsp ~ x_btsp, family = binomial)
## 
## Coefficients:
##             Estimate Std. Error z value Pr(>|z|)    
## (Intercept) -0.52119    0.41264  -1.263    0.207    
## x_btsp      -0.06812    0.01300  -5.240 1.61e-07 ***
## ---
## Signif. codes:  0 '***' 0.001 '**' 0.01 '*' 0.05 '.' 0.1 ' ' 1
## 
## (Dispersion parameter for binomial family taken to be 1)
## 
##     Null deviance: 138.589  on 99  degrees of freedom
## Residual deviance:  45.803  on 98  degrees of freedom
## AIC: 49.803
## 
## Number of Fisher Scoring iterations: 6
\end{verbatim}

\begin{Shaded}
\begin{Highlighting}[]
\FunctionTok{plot}\NormalTok{(x\_btsp, y\_btsp, }\AttributeTok{pch =} \DecValTok{19}\NormalTok{, }\AttributeTok{col =} \FunctionTok{ifelse}\NormalTok{(y\_btsp }\SpecialCharTok{==} \DecValTok{1}\NormalTok{, }\StringTok{"blue"}\NormalTok{, }\StringTok{"red"}\NormalTok{),}
     \AttributeTok{main =} \StringTok{"Logistic Regression Fit (Bootstrap)"}\NormalTok{, }\AttributeTok{xlab =} \StringTok{"x = Auto Time {-} Bus Time"}\NormalTok{, }\AttributeTok{ylab =} \StringTok{"P(car)"}\NormalTok{)}

\NormalTok{x\_seq\_btsp }\OtherTok{\textless{}{-}} \FunctionTok{seq}\NormalTok{(}\FunctionTok{min}\NormalTok{(x\_btsp), }\FunctionTok{max}\NormalTok{(x\_btsp), }\AttributeTok{length.out =} \DecValTok{100}\NormalTok{)}
\FunctionTok{lines}\NormalTok{(x\_seq\_btsp, }\FunctionTok{predict}\NormalTok{(modelo\_btstrap, }\AttributeTok{newdata =} \FunctionTok{data.frame}\NormalTok{(}\AttributeTok{x\_btsp =}\NormalTok{ x\_seq\_btsp), }\AttributeTok{type =} \StringTok{"response"}\NormalTok{),}
      \AttributeTok{col =} \StringTok{"darkgreen"}\NormalTok{, }\AttributeTok{lwd =} \DecValTok{2}\NormalTok{)}

\FunctionTok{legend}\NormalTok{(}\StringTok{"bottomleft"}\NormalTok{, }\AttributeTok{legend =} \FunctionTok{c}\NormalTok{(}\StringTok{"Observado 1"}\NormalTok{, }\StringTok{"Observado 0"}\NormalTok{, }\StringTok{"Curva logística"}\NormalTok{),}
       \AttributeTok{col =} \FunctionTok{c}\NormalTok{(}\StringTok{"blue"}\NormalTok{, }\StringTok{"red"}\NormalTok{, }\StringTok{"darkgreen"}\NormalTok{), }\AttributeTok{pch =} \FunctionTok{c}\NormalTok{(}\DecValTok{19}\NormalTok{, }\DecValTok{19}\NormalTok{, }\ConstantTok{NA}\NormalTok{), }\AttributeTok{lty =} \FunctionTok{c}\NormalTok{(}\ConstantTok{NA}\NormalTok{, }\ConstantTok{NA}\NormalTok{, }\DecValTok{1}\NormalTok{), }\AttributeTok{lwd =} \DecValTok{2}\NormalTok{, }\AttributeTok{cex =} \FloatTok{0.8}\NormalTok{)}
\end{Highlighting}
\end{Shaded}

\includegraphics{regresion_logit_probit_con_apendices_limpios_files/figure-latex/unnamed-chunk-6-1.pdf}

\subsection{Conclusión General}\label{conclusiuxf3n-general}

Ambos modelos, logit y probit, son útiles para clasificación binaria. El
modelo logit tiene una curva más empinada en el centro, mientras que el
probit es más suave en los extremos. La técnica de bootstrap refuerza la
validez de los estimadores obtenidos.

\section{Apéndices: Documentos TeX
relevantes}\label{apuxe9ndices-documentos-tex-relevantes}

\subsection{Derivadas Regresion}\label{derivadas-regresion}


\subsection{Maalouf Logistic Regression
Deducciones}\label{maalouf-logistic-regression-deducciones}


\subsection{Maalouf Logistic Regression
Full}\label{maalouf-logistic-regression-full}


\subsection{Resumen Ai Ml Con Referencias
Compila}\label{resumen-ai-ml-con-referencias-compila}


\subsection{Resumen Ai Ml Con Referencias
Final}\label{resumen-ai-ml-con-referencias-final}


\subsection{Resumen Ai Ml Con Referencias
Tikz}\label{resumen-ai-ml-con-referencias-tikz}


\subsection{Resumen Ai Ml Con
Referencias}\label{resumen-ai-ml-con-referencias}

\subsection{Resumen Logistic Regression
Completo}\label{resumen-logistic-regression-completo}


\subsection{Resumen Logistic Regression
Formal}\label{resumen-logistic-regression-formal}


\section{Apéndices: Documentos TeX (versiones
limpias)}\label{apuxe9ndices-documentos-tex-versiones-limpias}

\subsection{Derivadas Regresion}\label{derivadas-regresion-1}



\subsection{Maalouf Logistic Regression
Deducciones}\label{maalouf-logistic-regression-deducciones-1}



\subsection{Maalouf Logistic Regression
Full}\label{maalouf-logistic-regression-full-1}



\subsection{Resumen Ai Ml Con Referencias
Compila}\label{resumen-ai-ml-con-referencias-compila-1}



\subsection{Resumen Ai Ml Con Referencias
Final}\label{resumen-ai-ml-con-referencias-final-1}



\subsection{Resumen Ai Ml Con Referencias
Tikz}\label{resumen-ai-ml-con-referencias-tikz-1}



\subsection{Resumen Ai Ml Con
Referencias}\label{resumen-ai-ml-con-referencias-1}



\subsection{Resumen Logistic Regression
Completo}\label{resumen-logistic-regression-completo-1}



\subsection{Resumen Logistic Regression
Formal}\label{resumen-logistic-regression-formal-1}



\end{document}
