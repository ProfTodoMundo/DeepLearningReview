
\documentclass[12pt]{report}
\usepackage[utf8]{inputenc}
\usepackage{amsmath, amssymb, amsfonts, graphicx, algorithm, algpseudocode, hyperref}
\usepackage{geometry}
\usepackage{fancyhdr}
\usepackage{lipsum}
\geometry{margin=1in}

% Portada
\title{\textbf{Resumen Matem\'atico Ampliado}} 
\author{Carlos \\ \textit{Compilado con ayuda de ChatGPT}} 
\date{\today}

\begin{document}

\maketitle

\begin{center}
    \Large
    Basado en el art\'iculo: \\ 
    \textit{Logistic Regression in Data Analysis: An Overview} \\ 
    Maher Maalouf, 2011
\end{center}

\tableofcontents
\newpage

\chapter{Introducci\'on}
La regresi\'on log\'istica (RL) es una herramienta estad\'istica fundamental en la clasificaci\'on binaria. Este documento presenta un resumen detallado del modelo log\'istico, con \'enfasis en las deducciones matem\'aticas del gradiente, el Hessiano y sus extensiones con regularizaci\'on. Adem\'as, se discuten algoritmos de optimizaci\'on y correcciones para eventos raros.

\chapter{Modelo de regresi\'on log\'istica}
...

\chapter{Derivadas: Gradiente y Hessiano}
...

\chapter{Regularizaci\'on Ridge (L2)}
...

\chapter{TR-IRLS (Trust Region IRLS)}
...

\chapter{Eventos raros y correcciones}
...

\chapter{Correcci\'on de Firth}
...

\chapter{Regla de decisi\'on}
...

\chapter*{Referencias}
\addcontentsline{toc}{chapter}{Referencias}
Maalouf, M. (2011). Logistic regression in data analysis: An overview. \\ 
King, G., \& Zeng, L. (2001). Logistic regression in rare events data. \\ 
Firth, D. (1993). Bias reduction of maximum likelihood estimates.

\end{document}
