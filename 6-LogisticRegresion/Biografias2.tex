%===========================================
\documentclass[12pt]{article}
%===========================================
\usepackage[utf8]{inputenc}
%\usepackage[margin=2.5in]{geometry}
\usepackage{amsmath,amssymb,amsthm,amsfonts}
\usepackage{hyperref}
\usepackage{hyperref}
\usepackage{fancyhdr}
\usepackage{titlesec}
\usepackage{listings}
\usepackage{graphicx,graphics}
\usepackage{multicol}
\usepackage{multirow}
\usepackage{color}
\usepackage{float} 
\usepackage{subfig}
\usepackage[figuresright]{rotating}
\usepackage{enumerate}
\usepackage{anysize} 
\usepackage{url}
\usepackage{imakeidx}
\usepackage[left=0.5in, right=0.5in, top=1in, bottom=1in]{geometry}

%===========================================
\title{BIOGRAFIAS \\
\textit{PENDIENTES}}
\author{Carlos}
\date{}
%===========================================
\newtheorem{Criterio}{Criterio}%[section]
\newtheorem{Sup}{Supuesto}%[section]
\newtheorem{Note}{Nota}%[section]
\newtheorem{Ejem}{Ejemplo}%[section]
%===========================================
\begin{document}
%===========================================
\maketitle
\tableofcontents
%===========================================


\section{Pierre-François Verhulst (1804--1849)}

Pierre-François Verhulst (Bruselas, 28 de octubre de 1804 – Bruselas, 15 de febrero de 1849) fue un matemático belga conocido principalmente por haber propuesto la \emph{ecuación logística} de crecimiento poblacional:contentReference[oaicite:0]{index=0}. Estudió filología clásica en Bruselas, pero posteriormente se orientó a las matemáticas en la Universidad de Gante, donde obtuvo el doctorado en 1825:contentReference[oaicite:1]{index=1}. A temprana edad ganó reconocimiento con un premio por un ensayo sobre cálculo de variaciones:contentReference[oaicite:2]{index=2}, mostrando ya su talento en el análisis matemático.

En su trayectoria profesional, Verhulst ocupó posiciones académicas destacadas. Tras la independencia de Bélgica (1830), trabajó en la Real Academia Militar y luego, en 1835, se convirtió en profesor de matemáticas en la recién fundada Universidad Libre de Bruselas:contentReference[oaicite:3]{index=3}. Su prestigio científico le valió la membresía en la Real Academia de Bélgica, a la cual fue elegido en 1841:contentReference[oaicite:4]{index=4}; eventualmente llegó a ser presidente de dicha academia en 1848:contentReference[oaicite:5]{index=5}. Verhulst también participó activamente en la vida pública belga: estuvo involucrado en la Revolución belga de 1830 y en la defensa contra la invasión holandesa de 1831, reflejando un compromiso cívico más allá de sus intereses científicos:contentReference[oaicite:6]{index=6}.

La principal contribución científica de Verhulst es la formulación de la \emph{curva logística} (o ecuación logística) como modelo matemático del crecimiento de poblaciones. Influenciado por las ideas de Adolphe Quetelet sobre estadística social y las teorías demográficas de Malthus, Verhulst propuso en 1838 una ley de crecimiento poblacional que incorpora un factor de saturación para reflejar la existencia de un límite ambiental:contentReference[oaicite:7]{index=7}. En su obra \emph{Note sur la loi que la population suit dans son accroissement} (1838) y un artículo más completo de 1845, introduce la ecuación diferencial logística, usualmente escrita como $\frac{dN}{dt} = r\,N\left(1 - \frac{N}{K}\right)$, donde $r$ es la tasa de crecimiento intrínseca y $K$ es la capacidad de carga del medio:contentReference[oaicite:8]{index=8}. Esta ecuación produce la clásica curva en forma de “S” que inicialmente crece de manera exponencial pero luego se enlentece al aproximarse $N$ al valor límite $K$, corrigiendo así las predicciones ilimitadas del modelo de Malthus:contentReference[oaicite:9]{index=9}. La idea de Verhulst pasó desapercibida durante décadas, pero fue redescubierta por otros investigadores (por ejemplo, Raymond Pearl y Lowell Reed en la década de 1920) y se convirtió en un pilar de la demografía y la ecología poblacional. Desde la década de 1970, la ecuación logística de Verhulst adquirió una relevancia adicional al demostrarse que, bajo ciertas condiciones iterativas (el llamado mapa logístico), puede exhibir comportamiento caótico, lo que la convirtió en un ejemplo paradigmático en la teoría del caos:contentReference[oaicite:10]{index=10}.

Otro aporte notable de Verhulst fue su trabajo en análisis matemático, específicamente en el campo de las \emph{funciones elípticas}. En 1841 publicó un \emph{Tratado sobre las funciones elípticas} que resolvió problemas abiertos en integrales elípticas que incluso grandes matemáticos como Siméon-Denis Poisson no habían logrado solucionar:contentReference[oaicite:11]{index=11}. Este tratado, muy bien recibido, consolidó su reputación matemática y fue uno de los factores que condujo a su incorporación unánime a la Academia Real de Bélgica:contentReference[oaicite:12]{index=12}. 

La influencia de Verhulst en su campo ha sido duradera. Su ecuación logística sentó las bases para el estudio cuantitativo de las poblaciones en biología y ciencias sociales, y hoy forma parte fundamental de los cursos de modelización en ecología y dinámica poblacional. Términos como “crecimiento logístico” o “curva de Verhulst” se usan ampliamente para describir procesos de crecimiento acotados por recursos:contentReference[oaicite:13]{index=13}. Asimismo, el legado intelectual de Verhulst trasciende su época: la simple pero profunda idea de introducir la no linealidad (saturación) en modelos de crecimiento inspiró desarrollos posteriores en estadística (por ejemplo, la función logística utilizada en regresión logística) y sistemas dinámicos (como ya se mencionó, en estudios de caos determinista). En resumen, Verhulst es recordado como un pionero de la demografía matemática y un destacado matemático del siglo XIX cuyas ideas prefiguraron avances científicos del siglo XX:contentReference[oaicite:14]{index=14}:contentReference[oaicite:15]{index=15}.

\section{Joseph Berkson (1899--1982)}

Joseph Berkson (Nueva York, 14 de mayo de 1899 – Rochester (Minnesota), 12 de septiembre de 1982) fue un estadístico y médico estadounidense que realizó importantes contribuciones a la bioestadística. Aunque inicialmente se formó en ciencias físicas y medicina, orientó su carrera hacia la estadística médica aplicada. Obtuvo una sólida formación académica multidisciplinaria: se graduó con una licenciatura (B.S.) en el City College de Nueva York en 1920, completó una maestría (A.M.) en la Universidad de Columbia en 1922, y posteriormente obtuvo el título de médico (M.D.) en 1927 en la Universidad Johns Hopkins, seguido de un doctorado científico (Sc.D.) en 1928 en la misma institución:contentReference[oaicite:16]{index=16}. Esta combinación inusual de títulos le otorgó un perfil único como \emph{médico-estadístico}.

Berkson desarrolló la mayor parte de su carrera en la Clínica Mayo (Rochester, Minnesota), donde ingresó en 1931 como investigador. En 1932 fue nombrado jefe (y fundador) de la División de Bioestadística (entonces llamada División de Biometría y Estadística Médica) de la Clínica Mayo, puesto que desempeñó hasta su jubilación en 1964:contentReference[oaicite:17]{index=17}. Durante esos años también sirvió como profesor de biometría en la Universidad de Minnesota, contribuyendo a la formación de nuevos bioestadísticos:contentReference[oaicite:18]{index=18}. Bajo su liderazgo, la Clínica Mayo se convirtió en una de las instituciones pioneras en incorporar métodos estadísticos rigurosos en la investigación médica y clínica. Además de su labor en Mayo, durante la Segunda Guerra Mundial Berkson sirvió en el Cuerpo Médico de la Fuerza Aérea de los Estados Unidos (alcanzando el rango de coronel), donde aplicó sus conocimientos estadísticos a problemas militares de salud. A lo largo de su carrera, fue reconocido por sus colegas: por ejemplo, fue elegido Miembro de la American Statistical Association (ASA) en 1940:contentReference[oaicite:19]{index=19} y, culminando su prestigio, ingresó a la Academia Nacional de Ciencias de Estados Unidos en 1979:contentReference[oaicite:20]{index=20}.

En cuanto a sus contribuciones científicas, Joseph Berkson es quizás más conocido por introducir y promover el uso de la \emph{regresión logística} en análisis biomédicos. En 1944 acuñó el término “\emph{logit}” para referirse al \emph{logarithmic unit} (unidad logarítmica de probabilidad), al aplicar la función logística como curva de dosis-respuesta en bioensayos. Su trabajo “Application of the logistic function to bio-assay” (1944) demostró las ventajas de usar la función logística (frente a la función probit basada en la curva normal) para modelar datos binarios en experimentos biológicos. Este aporte sentó una base terminológica y metodológica que sería fundamental décadas más tarde, cuando la regresión logística se convirtió en una herramienta estándar en estadística e investigación operativa. Berkson también realizó contribuciones teóricas importantes en el entendimiento de los sesgos en datos médicos: en 1946 describió el fenómeno hoy conocido como la \emph{paradoja de Berkson} (o sesgo de Berkson). Esta “paradoja” es un tipo de sesgo de selección que explicó por qué, en estudios retrospectivos de casos y controles en hospitales, puede aparecer una correlación espuria negativa entre dos enfermedades o condiciones coexistentes debido a la forma en que se eligen los sujetos del estudio (pacientes hospitalizados). La paradoja de Berkson destacó la importancia de considerar cuidadosamente los criterios de selección de muestras en estudios epidemiológicos, y sus enseñanzas siguen vigentes en la metodología estadística médica moderna. Además, Berkson introdujo en 1950 un modelo de errores en variables particular, ahora denominado \emph{error de Berkson}. A diferencia del modelo clásico de error (donde se asume que el error está en la variable dependiente), el modelo de error de Berkson asume que el error se encuentra en la variable independiente o en la exposición asignada, caso relevante, por ejemplo, en estudios donde a grupos de sujetos se les asigna un valor medio de exposición en lugar de medirla individualmente. Este modelo es ampliamente utilizado en análisis de errores de medición, especialmente en epidemiología y estudios de calibración, y es fundamental para entender cómo la imprecisión en variables explicativas puede no sesgar los estimadores de la misma forma que en el caso clásico.

La influencia de Joseph Berkson en la estadística y la medicina es ampliamente reconocida. Sus esfuerzos por incorporar métodos estadísticos en entornos clínicos contribuyeron a consolidar la \emph{bioestadística} como disciplina. El concepto de “logit” y la regresión logística que él impulsó se convirtieron en herramientas indispensables en numerosas áreas, desde la salud pública hasta las ciencias sociales, permitiendo modelar probabilísticamente eventos binarios (éxito/fracaso, presencia/ausencia) con gran flexibilidad. Su insistencia en el cuidado con la interpretación de datos clínicos (como lo ejemplifica la paradoja de Berkson) influyó en generaciones de estadísticos y epidemiólogos para diseñar estudios con menos sesgos. Berkson también participó activamente en debates científicos de su época; por ejemplo, fue un crítico temprano de ciertas inferencias sobre la relación entre tabaquismo y cáncer de pulmón basadas solo en estudios retrospectivos de hospitales, aplicando su análisis de sesgo de selección (aunque esa postura generó controversia):contentReference[oaicite:28]{index=28}. En el balance general, Joseph Berkson es recordado como un innovador y \emph{pensador estadístico} de primer orden, que tendió puentes entre la medicina y la estadística. Sus aportes metodológicos y conceptuales dejaron un legado perdurable en la forma en que se diseñan experimentos biomédicos y se analizan datos clínicos, reforzando la evidencia cuantitativa en la toma de decisiones en salud.




\end{document}