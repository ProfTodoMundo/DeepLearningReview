
\documentclass[12pt]{article}
\usepackage[utf8]{inputenc}
\usepackage[spanish]{babel}
\usepackage{amsmath, amssymb, graphicx, booktabs, caption}
\usepackage{geometry}
\usepackage{hyperref}
\geometry{margin=1in}
\usepackage{natbib}
\usepackage{setspace}
\usepackage{titlesec}
\usepackage{fancyhdr}
\usepackage{color}
\usepackage{float}

\title{Modelos de Regresión Aplicados a Datos Reales:\\Lineal, Logística y Logística Multivariada}
\author{Carlos Ernesto}
\date{\today}

\pagestyle{fancy}
\fancyhf{}
\rhead{Modelos de Regresión}
\lhead{Carlos Ernesto}
\rfoot{\thepage}

\begin{document}

\maketitle
\tableofcontents
\newpage

\section{Introducción}
En este documento se exploran distintos modelos de regresión aplicados a datos reales en dos áreas clave: ciencias médicas y ciencias sociales. Se abordan modelos lineales, logísticos y logísticos multivariados utilizando herramientas estadísticas implementadas en el lenguaje de programación R.

\section{Marco Teórico}
\subsection{Regresión Lineal}
Explicación de la regresión lineal simple, ecuación del modelo, supuestos y estimación por mínimos cuadrados.

\subsection{Regresión Logística}
Fundamentos de la regresión logística, odds, log-odds y la función logística. Aplicaciones cuando la variable dependiente es binaria.

\subsection{Regresión Logística Multivariada}
Extensión de la regresión logística incluyendo múltiples predictores. Evaluación de efectos individuales y conjuntos.

\section{Metodología}
\subsection{Fuentes de datos}
Descripción de las bases utilizadas: cáncer, enfermedad renal, divorcio y redes sociales.

\subsection{Tratamiento y limpieza de datos en R}
Explicación del preprocesamiento realizado, valores faltantes, variables transformadas.

\subsection{Ajuste de modelos y validación}
Técnicas para seleccionar variables, ajuste de modelos y evaluación con métricas.

\section{Resultados}
\subsection{Caso 1: Cáncer de mama (Regresión Logística)}
\subsection{Caso 2: Enfermedad renal crónica (Logística multivariada)}
\subsection{Caso 3: Divorcio vs edad del hijo (Regresión logística)}
\subsection{Caso 4: Redes sociales y estado civil (Modelado exploratorio)}

\section{Discusión}
Comparación entre modelos, interpretabilidad, utilidad práctica en cada área.

\section{Conclusiones}
Principales hallazgos, limitaciones del estudio y sugerencias para investigaciones futuras.

\section*{Referencias}
\bibliographystyle{apalike}
\bibliography{bibliografia}

\end{document}
