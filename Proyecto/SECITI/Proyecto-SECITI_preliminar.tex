\documentclass[12pt]{article}
\usepackage[utf8]{inputenc}
\usepackage[spanish]{babel}
\usepackage{amsmath, amssymb, amsfonts}
\usepackage{geometry}
\usepackage{graphicx}
\usepackage{hyperref}
\geometry{letterpaper, margin=2.5cm}

\title{Propuesta de Investigación 2025\\Integración de Aprendizaje Automático en Sistemas de Colas con Encuestas}
\author{Carlos Ernesto Mart\'inez Rodr\'iguez\\Universidad Autónoma de la Ciudad de México\\
Academia de Matem\'aticas, Colegio de Ciencia y Tecnolog\'ia\\
Plantel Casa Libertad}
\date{Abril 2025}

\begin{document}

\maketitle
\tableofcontents
\newpage
%------------------------------------------------------------------------------------------
\section{Carta de Postulación}
%------------------------------------------------------------------------------------------
\newpage

\noindent\textbf{Secretaría de Ciencia, Humanidades, Tecnologías e Innovación}

\noindent Presente

\vspace{1cm}

\noindent Por medio de la presente, la Universidad Autónoma de la Ciudad de México, a través de su representante legal, manifiesta su interés en participar en la convocatoria \textit{Ciencia Básica y de Frontera 2025} como institución beneficiaria del proyecto titulado \textbf{Aplicaci\'on de Aprendizaje Automático en los Sistemas de Visitas y Teor\'ia de Colas}, el cual será coordinado por el profesor \textit{Carlos Ernesto Martínez Rodríguez}, quien funge como Responsable Técnico.
\medskip

\noindent La institución se compromete a proporcionar los recursos humanos, administrativos y logísticos necesarios para el correcto desarrollo del proyecto, así como a garantizar el cumplimiento de los términos y condiciones establecidos en la convocatoria, incluyendo la normatividad aplicable.

\medskip

\noindent Asimismo, manifiesta que cuenta con registro vigente en el \textbf{Sistema Rizoma} y en el \textbf{RENIECYT} (1701716), y que no mantiene adeudos ni litigios con la \textbf{SECIHTI} ni con el extinto \textbf{CONAHCyT}.

\medskip

\noindent Sin más por el momento, y reiterando nuestro respaldo institucional, quedamos atentos a cualquier requerimiento adicional.


\medskip



\vspace{1cm}
\noindent Atentamente,\\[0.3cm]
\textbf{M. en C. Juan Carlos Aguilar Franco}\\
Representante Legal\\
Universidad Autónoma de la Ciudad de México.\\

\newpage

%------------------------------------------------------------------------------------------
\section{Resumen Ejecutivo}
%------------------------------------------------------------------------------------------
Esta propuesta de investigación tiene como finalidad desarrollar una metodología que integre a los sistemas de colas —incluyendo los llamados sistemas de visitas— con algoritmos de aprendizaje automático (Machine Learning), en particular el aprendizaje supervisado. Este enfoque busca optimizar políticas de servicio bajo condiciones de estacionareidad, permitiendo adaptabilidad y mejora continua en tiempo real. La investigación se estructura en tres etapas: modelado teórico con base en procesos de Markov estacionarios; implementación de algoritmos de aprendizaje supervisado y por refuerzo; y validación mediante simulación computacional en Python y R. La propuesta se integra en el campo de las matemáticas aplicadas y ciencia de datos, con aplicaciones directas en telecomunicaciones, logística y manufactura, así como en procesos de selección. El impacto potencial abarca tanto avances teóricos en teoría de colas como el desarrollo de herramientas prácticas para sistemas adaptativos. Se espera que los resultados incluyan publicaciones científicas, presentación de resultados en congresos, simposios o foros, y recomendaciones de políticas de servicio. Este proyecto se alinea completamente con los objetivos de la convocatoria de Ciencia Básica y de Frontera 2025, al fomentar investigación innovadora, interdisciplinaria y de impacto nacional.\footnote{\textit{1110 caracteres y 202 palabras}}


%------------------------------------------------------------------------------------------
\section{Objetivo General}
%------------------------------------------------------------------------------------------
Desarrollar un marco matemático-computacional que integre algoritmos de aprendizaje automático, particularmente aprendizaje por refuerzo, en modelos de sistemas de colas con extensión natural a los sistemas de visitas, con el fin de optimizar dinámicamente las políticas de servicio en entornos variables. Esta integración permitirá formular modelos estocásticos adaptativos, basados en cadenas y procesos de Markov, con la capacidad de ajustar sus decisiones en tiempo real y mejorar indicadores de desempeño tales como los tiempos de espera, tiempos de servicio y tiempos de salida del sistema. Este objetivo se alinea con la convocatoria al fomentar investigación de frontera que articula matemáticas aplicadas, ciencia de datos y teoría de colas, con aplicaciones en sectores estratégicos previamente señalados.\footnote{\textit{736 caracteres y 132 palabras}}

%------------------------------------------------------------------------------------------
\section{Objetivos Específicos}
%------------------------------------------------------------------------------------------
\begin{itemize}
  \item Desarrollar modelos estocásticos para sistema de visitas que integren decisiones basadas en algoritmos de aprendizaje automático, considerando condiciones de funcionamiento reales.
  
  \item Analizar teóricamente la estabilidad y convergencia de los modelos híbridos mediante técnicas de probabilidad y procesos estocásticos, en particular cadenas de Markov estacionarias y procesos de Markov estacionarios.
  
  \item Comparar cuantitativamente políticas tradicionales (cerrada, exhaustiva y k-limitada) con políticas aprendidas por algoritmos de Machine Learning, en distintos escenarios simulados.
  
  \item Implementar prototipos de simulación en Python/R y validar los resultados mediante pruebas en ambientes controlados, utilizando métricas de desempeño estandarizadas en el área.\footnote{\textit{775 caracteres y 141 palabras}}
\end{itemize}

%------------------------------------------------------------------------------------------
\section{Antecedentes}
%------------------------------------------------------------------------------------------

Los sistemas de espera modelan situaciones en las que usuarios arriban a una o más colas para recibir un servicio según una política establecida, como la regla \textit{first in, first out} (FIFO). Los tiempos entre llegadas y la capacidad de las colas —ya sea finita o infinita— son elementos centrales para caracterizar su comportamiento. A partir de estas características, se definen métricas de desempeño como el tiempo promedio de espera, atención y servicio, fundamentales para describir la eficiencia del sistema bajo condiciones de estacionareidad.

Una extensión natural de estos modelos son las redes de colas, en las que múltiples estaciones con servidores propios operan de forma simultánea. Estas redes permiten representar sistemas reales como supermercados, centros de atención telefónica o procesos industriales, donde los usuarios pueden circular entre diferentes puntos de servicio.

Además de las redes de colas tradicionales, los sistemas de visitas representan una configuración donde uno o varios servidores se desplazan entre diferentes colas. A diferencia de las redes clásicas, las estaciones pueden ser atendidas de manera secuencial bajo políticas cíclicas, aleatorias o deterministas. Este enfoque introduce el concepto de tiempo de traslado entre colas y da lugar a métricas específicas como la longitud promedio de la cola o el tiempo de servicio por estación. Las políticas de atención también varían: desde esquemas que solo atienden a quienes ya están en la cola al llegar el servidor, hasta modelos que incluyen a quienes llegan mientras el servicio está en curso.

Dentro de esta clase, los llamados \textit{polling systems} constituyen un caso paradigmático: un único servidor o varios servidores visitan múltiples colas siguiendo una política definida. Aunque este tipo de sistemas ha sido ampliamente estudiado desde un enfoque probabilístico, suelen operar con parámetros estocásticos.

En contraste, el aprendizaje automático —especialmente el aprendizaje por refuerzo— permite desarrollar políticas de decisión adaptativas, útiles en contextos dinámicos y no estacionarios. En años recientes se han explorado aplicaciones de aprendizaje automático (ML) para simular o mejorar sistemas de atención, particularmente en logística y salud. Sin embargo, estos trabajos a menudo emplean heurísticas empíricas sin un marco matemático formal que integre rigurosamente los principios de la teoría de colas con los algoritmos de ML.

Esta propuesta parte de la hipótesis de que es posible enriquecer los modelos estocásticos mediante decisiones aprendidas a partir de datos, manteniendo propiedades clave como la estabilidad, ergodicidad y convergencia. Se plantea, por tanto, construir un marco unificado que permita representar e implementar sistemas de colas dinámicos, informados por datos y optimizados de manera continua. Esta integración representa una contribución significativa en la frontera entre matemáticas aplicadas y ciencia de datos.\footnote{\textit{2927 caracteres y 498 palabras}}

%------------------------------------------------------------------------------------------
\section{Hipótesis o Preguntas de Investigación}
%------------------------------------------------------------------------------------------
\begin{itemize}
  \item \textbf{1)} Es posible construir un modelo mixto que combine herramientas de aprendizaje automático y teoría de colas, bajo condiciones que aseguren la estabilidad del sistema o la existencia de un régimen estacionario.

  \item \textbf{2)} Las medidas tradicionales de desempeño pueden ser superadas mediante algoritmos de aprendizaje supervisado, enfocados en estimar valores promedio de tiempos de servicio, tiempos de permanencia y tiempos de llegada al sistema.

  \item \textbf{3)} Es viable implementar estos algoritmos de aprendizaje automático en lenguajes de programación como R o Python para calcular métricas estándar en sistemas de colas o de visitas.
  
  \item \textbf{4)} Es posible  implementar estos algoritmos de aprendizaje automático considerando, al menos la política de servicio exhaustiva y cerrada.

  \item \textbf{5)} La metodología propuesta puede aplicarse de manera efectiva en sistemas reales como transporte, mecanismos de selección y telecomunicaciones.\footnote{\textit{950 caracteres y 161 palabras}}
\end{itemize}

%------------------------------------------------------------------------------------------
\section{Pertinencia de la Propuesta}
%------------------------------------------------------------------------------------------

El proyecto se alinea con los objetivos generales de la convocatoria al promover investigación de frontera con enfoque interdisciplinario. Integra matemáticas aplicadas, ciencia de datos y procesos estocásticos para resolver problemas reales en telecomunicaciones y logística. Aporta además al desarrollo de capacidades en áreas estratégicas emergentes. Aunque existe abundante literatura sobre sistemas de colas y visitas, los estudios que vinculan aprendizaje automático —como redes neuronales o aprendizaje por refuerzo— con teoría de colas siguen siendo escasos. Esta propuesta, por tanto, busca cubrir un vacío relevante mediante un marco formal que combine ambos enfoques y abra nuevas rutas de investigación en la modelación de sistemas dinámicos adaptativos.\footnote{\textit{798 caracteres y 139 palabras}}

%------------------------------------------------------------------------------------------
\section{Metodología}
%------------------------------------------------------------------------------------------
Se realizará una revisión exhaustiva de literatura reciente en teoría de colas, sistemas de visitas y técnicas de ciencia de datos, con énfasis en aprendizaje automático supervisado. A partir de ello, se desarrollarán modelos estocásticos basados en cadenas y procesos de Markov estacionarios. Se implementarán algoritmos de aprendizaje en Python y R, y se construirán simulaciones para evaluar políticas de servicio (exhaustiva, cerrada y limitada). Finalmente, se analizarán propiedades como estabilidad, convergencia y eficiencia mediante el cálculo de medidas de desempeño y estimaciones numéricas.\footnote{\textit{800 caracteres y 131 palabras}}

%------------------------------------------------------------------------------------------
\section{Resultados Esperados}
%------------------------------------------------------------------------------------------

Se propone desarrollar una metodología formal que mejore el desempeño de los sistemas de colas mediante la integración de técnicas de aprendizaje automático, con énfasis en el aprendizaje supervisado. Se elaborará un artículo científico para ser sometido a revisión en revistas especializadas en matemáticas aplicadas y ciencia de datos. Se construirán programas robustos en R y Python que permitan estimar métricas de desempeño en sistemas de colas y visitas bajo diferentes políticas de servicio. Además, se prevé la participación en eventos académicos para presentar tanto los avances como los resultados finales del proyecto. Finalmente, se aplicará la metodología desarrollada en al menos un sistema real, con el objetivo de validar su eficacia en un entorno práctico.\footnote{\textit{780 caracteres y 129 palabras}}

%------------------------------------------------------------------------------------------
\section{Factores de Riesgo y Estrategias}
%------------------------------------------------------------------------------------------
Los principales riesgos identificados son la complejidad matemática y computacional del modelo, así como la disponibilidad limitada de datos para simulación. Para mitigarlos, se comenzará con modelos simplificados, se realizarán pruebas incrementales y se emplearán datos sintéticos validados. Se contempla además una retroalimentación continua con expertos del área para validar supuestos y enfoques. En cuanto a las aplicaciones, se buscará contar con al menos un modelo funcional que sirva como prototipo para su posterior generalización. Un riesgo adicional es la dificultad para acceder a bibliografía especializada; en respuesta, se considerará la adquisición de fuentes clave mediante licencias digitales o redes institucionales de acceso académico.\footnote{\textit{774 caracteres y 127 palabras}}

%------------------------------------------------------------------------------------------
\section{Impacto Social}
%------------------------------------------------------------------------------------------
La investigación tiene un alto potencial para mejorar la eficiencia de sistemas de atención al público, como transporte, mecanismos de selección y asignación, telecomunicaciones y logística. La implementación de políticas adaptativas optimizadas mediante aprendizaje automático permitirá reducir tiempos de espera, disminuir costos operativos y mejorar la experiencia de los usuarios en servicios tanto públicos como privados. Asimismo, el uso de modelos estocásticos con integración de datos contribuirá a agilizar los tiempos de cómputo requeridos para estimar medidas de desempeño, permitiendo así una toma de decisiones más rápida y precisa. Estos avances pueden ser clave para mejorar procesos estratégicos en instituciones y empresas que dependen de la eficiencia operativa en entornos dinámicos.\footnote{\textit{792 caracteres y 129 palabras}}

%------------------------------------------------------------------------------------------
\section{Bibliografìa}
%------------------------------------------------------------------------------------------

\begin{enumerate}
  \item Asmussen, S. (1987). \textit{Applied Probability and Queues}. John Wiley and Sons.
  \item Boon, M. A. A., van der Mei, R. D., \& Winands, E. M. M. (2011). Applications of polling systems.
  \item Chen, M. S., \& Yen, H. W. (2011). Applications of machine learning on multi-queue message scheduling. \textit{Expert Systems with Applications, 38}(4), 3323–3335.
  \item Chocron, E., Cohen, I., \& Feigin, P. (2022). Delay prediction for managing multiclass service systems: An investigation of queueing theory and machine learning approaches. \textit{IEEE Transactions on Engineering Management, 71}, 4469–4479.
  \item Disney, R. L., Farrell, R. L., \& Morais, P. R. (1973). A characterization of M/G/1 queues with renewal departure processes. \textit{Management Science, 19}(11), 1222–1228.
  \item Efrosinin, D., Vishnevsky, V., Stepanova, N., \& Sztrik, J. (2025). Use cases of machine learning in queueing theory based on a GI/G/K system.
  \item Glynn, P. W. (2022). Queueing theory: Past, present, and future. \textit{Queueing Systems, 100}(3), 169–171.
  \item Kleinrock, L. (1975). \textit{Queueing Systems: Volume 1: Theory}. Wiley-Interscience.
  \item Kyritsis, A. I., \& Deriaz, M. (2019). A machine learning approach to waiting time prediction in queueing scenarios. In \textit{2019 Second International Conference on Artificial Intelligence for Industries (AI4I)} (pp. 17–21). IEEE.
  \item Levy, H., \& Sidi, M. (1990). Polling systems: Applications, modeling, and optimization. \textit{IEEE Transactions on Communications, 30}(10), 1750–1760.
  \item Raeis, M., Tizghadam, A., \& Leon-Garcia, A. (2021). Queue-learning: A reinforcement learning approach for providing quality of service. In \textit{Proceedings of the AAAI Conference on Artificial Intelligence, 35}(1), 461–468.
  \item Roubos, A. (2007). \textit{Polling Systems and Their Applications}. Vrije Universiteit Amsterdam.
  \item Semenova, O. V., \& Vishnevskii, V. M. (2006). Mathematical methods to study the polling systems. \textit{Automation and Remote Control, 67}(2), 173–220.
  \item Sidi, M., \& Levy, H. (1991). Polling systems with simultaneous arrivals. \textit{IEEE Transactions on Communications, 39}(6), 823–827.
  \item Sokolov, A., Semenova, O., \& Larionov, A. (2024). Examining the performance of a distributed system through the application of queuing theory. In \textit{Distributed Computer and Communication Networks: 26th International Conference, DCCN 2023, Revised Selected Papers} (Vol. 2129, p. 16). Springer Nature.
  \item Sutton, R. S., \& Barto, A. G. (2018). \textit{Reinforcement Learning: An Introduction} (2nd ed.). MIT Press.
  \item Takagi, H. (1988). Queueing analysis of polling models. \textit{ACM Computing Surveys, 20}(1), 5–28.
  \item Vishnevsky, V., \& Gorbunova, A. V. (2021). Application of machine learning methods to solving problems of queuing theory. In \textit{Information Technologies and Mathematical Modelling} (pp. 304–316). Springer.
  \item Vishnevsky, V., Semenova, O., \& Bui, D. T. (2021). Using a machine learning approach for analysis of polling systems with correlated arrivals. In \textit{Distributed Computer and Communication Networks} (pp. 336–345). Springer.
  \item Wang, H., \& Hart, B. (2025). Queueing theory-based spare parts prediction through machine learning. In \textit{2025 Annual Reliability and Maintainability Symposium (RAMS)} (pp. 1–7). IEEE.
  \item Zhou, Z., Wang, X., \& Zhang, C. (2020). Queueing theory meets deep learning.\footnote{\textit{3155 caracteres y 447 palabras}}
\end{enumerate}




\section{Plan de Trabajo por Etapas}

\subsection*{Etapa 1 — Año 2025 (Duración: 6 meses)}
\textbf{Descripción:} Durante esta primera fase se establecerán las bases conceptuales y matemáticas del proyecto. Se realizará una revisión exhaustiva de la literatura en teoría de colas, procesos estocásticos, y aprendizaje por refuerzo. Posteriormente, se formularán modelos base de polling systems con estructuras dinámicas, y se definirán formalmente las variables relevantes y métricas de desempeño. Se comenzará con la implementación de simulaciones iniciales de políticas clásicas (round-robin, gated, exhaustive) como referencia para análisis comparativo.

\textbf{Metas:}
\begin{itemize}
  \item Realizar revisión bibliográfica y sistematización de modelos existentes de colas y aprendizaje automático.
  \item Formular modelos matemáticos estocásticos básicos y definir métricas clave de desempeño.
\end{itemize}

\textbf{Entregables:}
\begin{itemize}
  \item Documento técnico con marco teórico y estado del arte.
  \item Prototipo de simulación inicial de políticas tradicionales en sistemas de colas.
\end{itemize}

\textbf{Actividades:}
\begin{itemize}
  \item Revisión de al menos 50 artículos clave sobre ML y teoría de colas. (97 caracteres)
  \item Modelado en papel y validación formal de condiciones de estabilidad. (91 caracteres)
  \item Implementación de modelos clásicos con Python/SimPy. (67 caracteres)
  \item Redacción del informe técnico de avance. (49 caracteres)
\end{itemize}

\subsection*{Etapa 2 — Año 2026 (Duración: 6 meses)}
\textbf{Descripción:} Esta fase se centrará en la implementación computacional de los modelos propuestos y en el desarrollo de algoritmos de aprendizaje automático, particularmente aprendizaje por refuerzo. Se diseñarán entornos de simulación para probar políticas de servicio aprendidas dinámicamente y se compararán con políticas tradicionales. También se trabajará en la recolección de datos simulados y en la evaluación del rendimiento de los modelos mediante métricas cuantitativas.

\textbf{Metas:}
\begin{itemize}
  \item Desarrollar e implementar algoritmos de ML adaptados a sistemas de colas.
  \item Diseñar un entorno de simulación robusto para pruebas comparativas.
\end{itemize}

\textbf{Entregables:}
\begin{itemize}
  \item Código funcional de algoritmos de aprendizaje por refuerzo.
  \item Informe de simulación con análisis comparativo y visualizaciones.
\end{itemize}

\textbf{Actividades:}
\begin{itemize}
  \item Codificación de entorno de simulación en Python con ML. (61 caracteres)
  \item Integración de modelos de aprendizaje tipo Q-Learning. (59 caracteres)
  \item Generación de conjuntos de datos sintéticos controlados. (58 caracteres)
  \item Redacción del informe de resultados y visualizaciones. (56 caracteres)
\end{itemize}

\subsection*{Etapa 3 — Año 2027 (Duración: 6 meses)}
\textbf{Descripción:} La última etapa estará enfocada en la validación teórica y práctica de los resultados obtenidos. Se realizarán análisis de estabilidad, convergencia y desempeño. Se prepararán artículos científicos para su publicación en revistas de alto impacto. También se desarrollará una herramienta de visualización interactiva y se socializarán los hallazgos con la comunidad académica.

\textbf{Metas:}
\begin{itemize}
  \item Validar matemáticamente los modelos desarrollados.
  \item Publicar y difundir los resultados científicos y técnicos del proyecto.
\end{itemize}

\textbf{Entregables:}
\begin{itemize}
  \item Artículo científico sometido a revista indexada.
  \item Herramienta de visualización y reporte final del proyecto.
\end{itemize}

\textbf{Actividades:}
\begin{itemize}
  \item Evaluación formal de condiciones de estabilidad. (52 caracteres)
  \item Envío de artículo a revista científica en matemáticas aplicadas. (68 caracteres)
  \item Diseño y documentación de herramienta visual en Shiny o Dash. (64 caracteres)
  \item Presentación de resultados en coloquios y congresos. (52 caracteres)
\end{itemize}
\section{Plan de Trabajo por Etapas}

\subsection*{Etapa 1 — Año 2025}
\textbf{Duración:} 6 meses\\
\textbf{Descripción:} Durante esta primera fase se establecerán las bases conceptuales y matemáticas del proyecto. Se realizará una revisión exhaustiva de la literatura en teoría de colas, procesos estocásticos, y aprendizaje por refuerzo. Posteriormente, se formularán modelos base de polling systems con estructuras dinámicas, y se definirán formalmente las variables relevantes y métricas de desempeño. Se comenzará con la implementación de simulaciones iniciales de políticas clásicas (round-robin, gated, exhaustive) como referencia para análisis comparativo.

\textbf{Metas:}
\begin{itemize}
  \item Realizar revisión bibliográfica y sistematización de modelos existentes de colas y aprendizaje automático.
  \item Formular modelos matemáticos estocásticos básicos y definir métricas clave de desempeño.
\end{itemize}

\textbf{Entregables:}
\begin{itemize}
  \item Documento técnico con marco teórico y estado del arte.
  \item Prototipo de simulación inicial de políticas tradicionales en sistemas de colas.
\end{itemize}

\textbf{Actividades:}
\begin{itemize}
  \item Revisión de al menos 50 artículos clave sobre ML y teoría de colas.
  \item Modelado en papel y validación formal de condiciones de estabilidad.
  \item Implementación de modelos clásicos con Python/SimPy.
  \item Redacción del informe técnico de avance.
\end{itemize}

\subsection*{Etapa 2 — Año 2026}
\textbf{Duración:} 12 meses\\
\textbf{Descripción:} En esta segunda fase se desarrollarán e implementarán algoritmos de aprendizaje automático, en particular aprendizaje por refuerzo, aplicados a los modelos de colas formulados. Se evaluarán distintas arquitecturas de agentes de decisión adaptativa, midiendo su desempeño contra políticas tradicionales en diversos escenarios simulados. También se realizarán análisis matemáticos sobre estabilidad, convergencia y ergodicidad de los modelos híbridos. Esta etapa busca consolidar la parte computacional y teórica del proyecto mediante simulaciones robustas.

\textbf{Metas:}
\begin{itemize}
  \item Desarrollar e implementar algoritmos de ML aplicados a los modelos de colas.
  \item Analizar y validar el comportamiento de los modelos híbridos mediante pruebas computacionales.
\end{itemize}

\textbf{Entregables:}
\begin{itemize}
  \item Repositorio de código con políticas de ML implementadas.
  \item Informe técnico de análisis comparativo de políticas.
\end{itemize}

\textbf{Actividades:}
\begin{itemize}
  \item Diseño de entorno de simulación para aprendizaje por refuerzo.
  \item Implementación de políticas con TensorFlow/Keras.
  \item Ejecución de simulaciones comparativas en escenarios variables.
  \item Análisis de resultados y validación con métricas de desempeño.
\end{itemize}

\subsection*{Etapa 3 — Año 2027}
\textbf{Duración:} 12 meses\\
\textbf{Descripción:} La etapa final se enfocará en la validación empírica de los modelos desarrollados y la preparación de publicaciones científicas. Se realizarán ajustes al modelo con base en los resultados obtenidos y se elaborarán herramientas de visualización de resultados. Además, se integrará un informe final con recomendaciones, implicaciones teóricas y aplicadas, así como posibles rutas para aplicaciones futuras en sectores estratégicos.

\textbf{Metas:}
\begin{itemize}
  \item Validar los modelos en entornos simulados y ajustar parámetros finales.
  \item Difundir los resultados a través de publicaciones, presentaciones y herramientas digitales.
\end{itemize}

\textbf{Entregables:}
\begin{itemize}
  \item Artículo científico para revista indexada.
  \item Informe final y repositorio con visualización de resultados.
\end{itemize}

\textbf{Actividades:}
\begin{itemize}
  \item Ajuste final de hiperparámetros y simulaciones.
  \item Redacción y envío de artículo a revista académica.
  \item Generación de panel de visualización de políticas y desempeño.
  \item Entrega del informe final y cierre del proyecto.
\end{itemize}

\section{Plan de Trabajo por Etapas}

\subsection*{Etapa 1 — Año 2025 (Duración: 6 meses)}
\textbf{Descripción:} Durante esta primera fase se establecerán las bases conceptuales y matemáticas del proyecto. Se realizará una revisión exhaustiva de la literatura en teoría de colas, procesos estocásticos, y aprendizaje por refuerzo. Posteriormente, se formularán modelos base de polling systems con estructuras dinámicas, y se definirán formalmente las variables relevantes y métricas de desempeño. Se comenzará con la implementación de simulaciones iniciales de políticas clásicas (round-robin, gated, exhaustive) como referencia para análisis comparativo.

\textbf{Metas:}
\begin{itemize}
  \item Realizar revisión bibliográfica y sistematización de modelos existentes de colas y aprendizaje automático.
  \item Formular modelos matemáticos estocásticos básicos y definir métricas clave de desempeño.
\end{itemize}

\textbf{Entregables:}
\begin{itemize}
  \item Documento técnico con marco teórico y estado del arte.
  \item Prototipo de simulación inicial de políticas tradicionales en sistemas de colas.
\end{itemize}

\textbf{Actividades:}
\begin{itemize}
  \item Revisión de al menos 50 artículos clave sobre ML y teoría de colas. (97 caracteres)
  \item Modelado en papel y validación formal de condiciones de estabilidad. (91 caracteres)
  \item Implementación de modelos clásicos con Python/SimPy. (67 caracteres)
  \item Redacción del informe técnico de avance. (49 caracteres)
\end{itemize}

\subsection*{Etapa 2 — Año 2026 (Duración: 6 meses)}
\textbf{Descripción:} Esta fase se centrará en la implementación computacional de los modelos propuestos y en el desarrollo de algoritmos de aprendizaje automático, particularmente aprendizaje por refuerzo. Se diseñarán entornos de simulación para probar políticas de servicio aprendidas dinámicamente y se compararán con políticas tradicionales. También se trabajará en la recolección de datos simulados y en la evaluación del rendimiento de los modelos mediante métricas cuantitativas.

\textbf{Metas:}
\begin{itemize}
  \item Desarrollar e implementar algoritmos de ML adaptados a sistemas de colas.
  \item Diseñar un entorno de simulación robusto para pruebas comparativas.
\end{itemize}

\textbf{Entregables:}
\begin{itemize}
  \item Código funcional de algoritmos de aprendizaje por refuerzo.
  \item Informe de simulación con análisis comparativo y visualizaciones.
\end{itemize}

\textbf{Actividades:}
\begin{itemize}
  \item Codificación de entorno de simulación en Python con ML. (61 caracteres)
  \item Integración de modelos de aprendizaje tipo Q-Learning. (59 caracteres)
  \item Generación de conjuntos de datos sintéticos controlados. (58 caracteres)
  \item Redacción del informe de resultados y visualizaciones. (56 caracteres)
\end{itemize}

\subsection*{Etapa 3 — Año 2027 (Duración: 6 meses)}
\textbf{Descripción:} La última etapa estará enfocada en la validación teórica y práctica de los resultados obtenidos. Se realizarán análisis de estabilidad, convergencia y desempeño. Se prepararán artículos científicos para su publicación en revistas de alto impacto. También se desarrollará una herramienta de visualización interactiva y se socializarán los hallazgos con la comunidad académica.

\textbf{Metas:}
\begin{itemize}
  \item Validar matemáticamente los modelos desarrollados.
  \item Publicar y difundir los resultados científicos y técnicos del proyecto.
\end{itemize}

\textbf{Entregables:}
\begin{itemize}
  \item Artículo científico sometido a revista indexada.
  \item Herramienta de visualización y reporte final del proyecto.
\end{itemize}

\textbf{Actividades:}
\begin{itemize}
  \item Evaluación formal de condiciones de estabilidad. (52 caracteres)
  \item Envío de artículo a revista científica en matemáticas aplicadas. (68 caracteres)
  \item Diseño y documentación de herramienta visual en Shiny o Dash. (64 caracteres)
  \item Presentación de resultados en coloquios y congresos. (52 caracteres)
\end{itemize}



\section{Plan de Trabajo por Etapas}

\subsection*{Etapa 1 — Año 2025 (Duración: 6 meses)}
\textbf{Descripción:} Durante esta primera fase se establecerán las bases conceptuales y matemáticas del proyecto. Se realizará una revisión exhaustiva de la literatura en teoría de colas, procesos estocásticos, y aprendizaje por refuerzo. Posteriormente, se formularán modelos base de polling systems con estructuras dinámicas, y se definirán formalmente las variables relevantes y métricas de desempeño. Se comenzará con la implementación de simulaciones iniciales de políticas clásicas (round-robin, gated, exhaustive) como referencia para análisis comparativo.

\textbf{Metas:}
\begin{itemize}
  \item Realizar revisión bibliográfica y sistematización de modelos existentes de colas y aprendizaje automático.
  \item Formular modelos matemáticos estocásticos básicos y definir métricas clave de desempeño.
\end{itemize}

\textbf{Entregables:}
\begin{itemize}
  \item Documento técnico con marco teórico y estado del arte.
  \item Prototipo de simulación inicial de políticas tradicionales en sistemas de colas.
\end{itemize}

\textbf{Actividades:}
\begin{itemize}
  \item Revisión de al menos 50 artículos clave sobre ML y teoría de colas. (97 caracteres)
  \item Modelado en papel y validación formal de condiciones de estabilidad. (91 caracteres)
  \item Implementación de modelos clásicos con Python/SimPy. (67 caracteres)
  \item Redacción del informe técnico de avance. (49 caracteres)
\end{itemize}

\subsection*{Etapa 2 — Año 2026 (Duración: 6 meses)}
\textbf{Descripción:} Esta fase se centrará en la implementación computacional de los modelos propuestos y en el desarrollo de algoritmos de aprendizaje automático, particularmente aprendizaje por refuerzo. Se diseñarán entornos de simulación para probar políticas de servicio aprendidas dinámicamente y se compararán con políticas tradicionales. También se trabajará en la recolección de datos simulados y en la evaluación del rendimiento de los modelos mediante métricas cuantitativas.

\textbf{Metas:}
\begin{itemize}
  \item Desarrollar e implementar algoritmos de ML adaptados a sistemas de colas.
  \item Diseñar un entorno de simulación robusto para pruebas comparativas.
\end{itemize}

\textbf{Entregables:}
\begin{itemize}
  \item Código funcional de algoritmos de aprendizaje por refuerzo.
  \item Informe de simulación con análisis comparativo y visualizaciones.
\end{itemize}

\textbf{Actividades:}
\begin{itemize}
  \item Codificación de entorno de simulación en Python con ML. (61 caracteres)
  \item Integración de modelos de aprendizaje tipo Q-Learning. (59 caracteres)
  \item Generación de conjuntos de datos sintéticos controlados. (58 caracteres)
  \item Redacción del informe de resultados y visualizaciones. (56 caracteres)
\end{itemize}

\subsection*{Etapa 3 — Año 2027 (Duración: 6 meses)}
\textbf{Descripción:} La última etapa estará enfocada en la validación teórica y práctica de los resultados obtenidos. Se realizarán análisis de estabilidad, convergencia y desempeño. Se prepararán artículos científicos para su publicación en revistas de alto impacto. También se desarrollará una herramienta de visualización interactiva y se socializarán los hallazgos con la comunidad académica.

\textbf{Metas:}
\begin{itemize}
  \item Validar matemáticamente los modelos desarrollados.
  \item Publicar y difundir los resultados científicos y técnicos del proyecto.
\end{itemize}

\textbf{Entregables:}
\begin{itemize}
  \item Artículo científico sometido a revista indexada.
  \item Herramienta de visualización y reporte final del proyecto.
\end{itemize}

\textbf{Actividades:}
\begin{itemize}
  \item Evaluación formal de condiciones de estabilidad. (52 caracteres)
  \item Envío de artículo a revista científica en matemáticas aplicadas. (68 caracteres)
  \item Diseño y documentación de herramienta visual en Shiny o Dash. (64 caracteres)
  \item Presentación de resultados en coloquios y congresos. (52 caracteres)
\end{itemize}
\section{Plan de Trabajo por Etapas}

\subsection*{Etapa 1 — Año 2025}
\textbf{Duración:} 6 meses\\
\textbf{Descripción:} Durante esta primera fase se establecerán las bases conceptuales y matemáticas del proyecto. Se realizará una revisión exhaustiva de la literatura en teoría de colas, procesos estocásticos, y aprendizaje por refuerzo. Posteriormente, se formularán modelos base de polling systems con estructuras dinámicas, y se definirán formalmente las variables relevantes y métricas de desempeño. Se comenzará con la implementación de simulaciones iniciales de políticas clásicas (round-robin, gated, exhaustive) como referencia para análisis comparativo.

\textbf{Metas:}
\begin{itemize}
  \item Realizar revisión bibliográfica y sistematización de modelos existentes de colas y aprendizaje automático.
  \item Formular modelos matemáticos estocásticos básicos y definir métricas clave de desempeño.
\end{itemize}

\textbf{Entregables:}
\begin{itemize}
  \item Documento técnico con marco teórico y estado del arte.
  \item Prototipo de simulación inicial de políticas tradicionales en sistemas de colas.
\end{itemize}

\textbf{Actividades:}
\begin{itemize}
  \item Revisión de al menos 50 artículos clave sobre ML y teoría de colas.
  \item Modelado en papel y validación formal de condiciones de estabilidad.
  \item Implementación de modelos clásicos con Python/SimPy.
  \item Redacción del informe técnico de avance.
\end{itemize}

\subsection*{Etapa 2 — Año 2026}
\textbf{Duración:} 12 meses\\
\textbf{Descripción:} En esta segunda fase se desarrollarán e implementarán algoritmos de aprendizaje automático, en particular aprendizaje por refuerzo, aplicados a los modelos de colas formulados. Se evaluarán distintas arquitecturas de agentes de decisión adaptativa, midiendo su desempeño contra políticas tradicionales en diversos escenarios simulados. También se realizarán análisis matemáticos sobre estabilidad, convergencia y ergodicidad de los modelos híbridos. Esta etapa busca consolidar la parte computacional y teórica del proyecto mediante simulaciones robustas.

\textbf{Metas:}
\begin{itemize}
  \item Desarrollar e implementar algoritmos de ML aplicados a los modelos de colas.
  \item Analizar y validar el comportamiento de los modelos híbridos mediante pruebas computacionales.
\end{itemize}

\textbf{Entregables:}
\begin{itemize}
  \item Repositorio de código con políticas de ML implementadas.
  \item Informe técnico de análisis comparativo de políticas.
\end{itemize}

\textbf{Actividades:}
\begin{itemize}
  \item Diseño de entorno de simulación para aprendizaje por refuerzo.
  \item Implementación de políticas con TensorFlow/Keras.
  \item Ejecución de simulaciones comparativas en escenarios variables.
  \item Análisis de resultados y validación con métricas de desempeño.
\end{itemize}

\subsection*{Etapa 3 — Año 2027}
\textbf{Duración:} 12 meses\\
\textbf{Descripción:} La etapa final se enfocará en la validación empírica de los modelos desarrollados y la preparación de publicaciones científicas. Se realizarán ajustes al modelo con base en los resultados obtenidos y se elaborarán herramientas de visualización de resultados. Además, se integrará un informe final con recomendaciones, implicaciones teóricas y aplicadas, así como posibles rutas para aplicaciones futuras en sectores estratégicos.

\textbf{Metas:}
\begin{itemize}
  \item Validar los modelos en entornos simulados y ajustar parámetros finales.
  \item Difundir los resultados a través de publicaciones, presentaciones y herramientas digitales.
\end{itemize}

\textbf{Entregables:}
\begin{itemize}
  \item Artículo científico para revista indexada.
  \item Informe final y repositorio con visualización de resultados.
\end{itemize}

\textbf{Actividades:}
\begin{itemize}
  \item Ajuste final de hiperparámetros y simulaciones.
  \item Redacción y envío de artículo a revista académica.
  \item Generación de panel de visualización de políticas y desempeño.
  \item Entrega del informe final y cierre del proyecto.
\end{itemize}

\section{Plan de Trabajo por Etapas}

\subsection*{Etapa 1 — Año 2025 (Duración: 6 meses)}
\textbf{Descripción:} Durante esta primera fase se establecerán las bases conceptuales y matemáticas del proyecto. Se realizará una revisión exhaustiva de la literatura en teoría de colas, procesos estocásticos, y aprendizaje por refuerzo. Posteriormente, se formularán modelos base de polling systems con estructuras dinámicas, y se definirán formalmente las variables relevantes y métricas de desempeño. Se comenzará con la implementación de simulaciones iniciales de políticas clásicas (round-robin, gated, exhaustive) como referencia para análisis comparativo.

\textbf{Metas:}
\begin{itemize}
  \item Realizar revisión bibliográfica y sistematización de modelos existentes de colas y aprendizaje automático.
  \item Formular modelos matemáticos estocásticos básicos y definir métricas clave de desempeño.
\end{itemize}

\textbf{Entregables:}
\begin{itemize}
  \item Documento técnico con marco teórico y estado del arte.
  \item Prototipo de simulación inicial de políticas tradicionales en sistemas de colas.
\end{itemize}

\textbf{Actividades:}
\begin{itemize}
  \item Revisión de al menos 50 artículos clave sobre ML y teoría de colas. (97 caracteres)
  \item Modelado en papel y validación formal de condiciones de estabilidad. (91 caracteres)
  \item Implementación de modelos clásicos con Python/SimPy. (67 caracteres)
  \item Redacción del informe técnico de avance. (49 caracteres)
\end{itemize}

\subsection*{Etapa 2 — Año 2026 (Duración: 6 meses)}
\textbf{Descripción:} Esta fase se centrará en la implementación computacional de los modelos propuestos y en el desarrollo de algoritmos de aprendizaje automático, particularmente aprendizaje por refuerzo. Se diseñarán entornos de simulación para probar políticas de servicio aprendidas dinámicamente y se compararán con políticas tradicionales. También se trabajará en la recolección de datos simulados y en la evaluación del rendimiento de los modelos mediante métricas cuantitativas.

\textbf{Metas:}
\begin{itemize}
  \item Desarrollar e implementar algoritmos de ML adaptados a sistemas de colas.
  \item Diseñar un entorno de simulación robusto para pruebas comparativas.
\end{itemize}

\textbf{Entregables:}
\begin{itemize}
  \item Código funcional de algoritmos de aprendizaje por refuerzo.
  \item Informe de simulación con análisis comparativo y visualizaciones.
\end{itemize}

\textbf{Actividades:}
\begin{itemize}
  \item Codificación de entorno de simulación en Python con ML. (61 caracteres)
  \item Integración de modelos de aprendizaje tipo Q-Learning. (59 caracteres)
  \item Generación de conjuntos de datos sintéticos controlados. (58 caracteres)
  \item Redacción del informe de resultados y visualizaciones. (56 caracteres)
\end{itemize}

\subsection*{Etapa 3 — Año 2027 (Duración: 6 meses)}
\textbf{Descripción:} La última etapa estará enfocada en la validación teórica y práctica de los resultados obtenidos. Se realizarán análisis de estabilidad, convergencia y desempeño. Se prepararán artículos científicos para su publicación en revistas de alto impacto. También se desarrollará una herramienta de visualización interactiva y se socializarán los hallazgos con la comunidad académica.

\textbf{Metas:}
\begin{itemize}
  \item Validar matemáticamente los modelos desarrollados.
  \item Publicar y difundir los resultados científicos y técnicos del proyecto.
\end{itemize}

\textbf{Entregables:}
\begin{itemize}
  \item Artículo científico sometido a revista indexada.
  \item Herramienta de visualización y reporte final del proyecto.
\end{itemize}

\textbf{Actividades:}
\begin{itemize}
  \item Evaluación formal de condiciones de estabilidad. (52 caracteres)
  \item Envío de artículo a revista científica en matemáticas aplicadas. (68 caracteres)
  \item Diseño y documentación de herramienta visual en Shiny o Dash. (64 caracteres)
  \item Presentación de resultados en coloquios y congresos. (52 caracteres)
\end{itemize}
\section{Plan de Trabajo por Etapas}

\subsection*{Etapa 1 — Año 2025}
\textbf{Duración:} 6 meses\\
\textbf{Descripción:} Durante esta primera fase se establecerán las bases conceptuales y matemáticas del proyecto. Se realizará una revisión exhaustiva de la literatura en teoría de colas, procesos estocásticos, y aprendizaje por refuerzo. Posteriormente, se formularán modelos base de polling systems con estructuras dinámicas, y se definirán formalmente las variables relevantes y métricas de desempeño. Se comenzará con la implementación de simulaciones iniciales de políticas clásicas (round-robin, gated, exhaustive) como referencia para análisis comparativo.

\textbf{Metas:}
\begin{itemize}
  \item Realizar revisión bibliográfica y sistematización de modelos existentes de colas y aprendizaje automático.
  \item Formular modelos matemáticos estocásticos básicos y definir métricas clave de desempeño.
\end{itemize}

\textbf{Entregables:}
\begin{itemize}
  \item Documento técnico con marco teórico y estado del arte.
  \item Prototipo de simulación inicial de políticas tradicionales en sistemas de colas.
\end{itemize}

\textbf{Actividades:}
\begin{itemize}
  \item Revisión de al menos 50 artículos clave sobre ML y teoría de colas.
  \item Modelado en papel y validación formal de condiciones de estabilidad.
  \item Implementación de modelos clásicos con Python/SimPy.
  \item Redacción del informe técnico de avance.
\end{itemize}

\subsection*{Etapa 2 — Año 2026}
\textbf{Duración:} 12 meses\\
\textbf{Descripción:} En esta segunda fase se desarrollarán e implementarán algoritmos de aprendizaje automático, en particular aprendizaje por refuerzo, aplicados a los modelos de colas formulados. Se evaluarán distintas arquitecturas de agentes de decisión adaptativa, midiendo su desempeño contra políticas tradicionales en diversos escenarios simulados. También se realizarán análisis matemáticos sobre estabilidad, convergencia y ergodicidad de los modelos híbridos. Esta etapa busca consolidar la parte computacional y teórica del proyecto mediante simulaciones robustas.

\textbf{Metas:}
\begin{itemize}
  \item Desarrollar e implementar algoritmos de ML aplicados a los modelos de colas.
  \item Analizar y validar el comportamiento de los modelos híbridos mediante pruebas computacionales.
\end{itemize}

\textbf{Entregables:}
\begin{itemize}
  \item Repositorio de código con políticas de ML implementadas.
  \item Informe técnico de análisis comparativo de políticas.
\end{itemize}

\textbf{Actividades:}
\begin{itemize}
  \item Diseño de entorno de simulación para aprendizaje por refuerzo.
  \item Implementación de políticas con TensorFlow/Keras.
  \item Ejecución de simulaciones comparativas en escenarios variables.
  \item Análisis de resultados y validación con métricas de desempeño.
\end{itemize}

\subsection*{Etapa 3 — Año 2027}
\textbf{Duración:} 12 meses\\
\textbf{Descripción:} La etapa final se enfocará en la validación empírica de los modelos desarrollados y la preparación de publicaciones científicas. Se realizarán ajustes al modelo con base en los resultados obtenidos y se elaborarán herramientas de visualización de resultados. Además, se integrará un informe final con recomendaciones, implicaciones teóricas y aplicadas, así como posibles rutas para aplicaciones futuras en sectores estratégicos.

\textbf{Metas:}
\begin{itemize}
  \item Validar los modelos en entornos simulados y ajustar parámetros finales.
  \item Difundir los resultados a través de publicaciones, presentaciones y herramientas digitales.
\end{itemize}

\textbf{Entregables:}
\begin{itemize}
  \item Artículo científico para revista indexada.
  \item Informe final y repositorio con visualización de resultados.
\end{itemize}

\textbf{Actividades:}
\begin{itemize}
  \item Ajuste final de hiperparámetros y simulaciones.
  \item Redacción y envío de artículo a revista académica.
  \item Generación de panel de visualización de políticas y desempeño.
  \item Entrega del informe final y cierre del proyecto.
\end{itemize}


\section{Plan de Trabajo por Etapas}

\subsection*{Etapa 1 — Año 2025}
\textbf{Duración:} 6 meses\\
\textbf{Descripción:} Durante esta primera fase se establecerán las bases conceptuales y matemáticas del proyecto. Se realizará una revisión exhaustiva de la literatura en teoría de colas, procesos estocásticos, y aprendizaje por refuerzo. Posteriormente, se formularán modelos base de polling systems con estructuras dinámicas, y se definirán formalmente las variables relevantes y métricas de desempeño. Se comenzará con la implementación de simulaciones iniciales de políticas clásicas (round-robin, gated, exhaustive) como referencia para análisis comparativo.

\textbf{Metas:}
\begin{itemize}
  \item Realizar revisión bibliográfica y sistematización de modelos existentes de colas y aprendizaje automático.
  \item Formular modelos matemáticos estocásticos básicos y definir métricas clave de desempeño.
\end{itemize}

\textbf{Entregables:}
\begin{itemize}
  \item Documento técnico con marco teórico y estado del arte.
  \item Prototipo de simulación inicial de políticas tradicionales en sistemas de colas.
\end{itemize}

\textbf{Actividades:}
\begin{itemize}
  \item Revisión de al menos 50 artículos clave sobre ML y teoría de colas.
  \item Modelado en papel y validación formal de condiciones de estabilidad.
  \item Implementación de modelos clásicos con Python/SimPy.
  \item Redacción del informe técnico de avance.
\end{itemize}

\subsection*{Etapa 2 — Año 2026}
\textbf{Duración:} 12 meses\\
\textbf{Descripción:} En esta segunda fase se desarrollarán e implementarán algoritmos de aprendizaje automático, en particular aprendizaje por refuerzo, aplicados a los modelos de colas formulados. Se evaluarán distintas arquitecturas de agentes de decisión adaptativa, midiendo su desempeño contra políticas tradicionales en diversos escenarios simulados. También se realizarán análisis matemáticos sobre estabilidad, convergencia y ergodicidad de los modelos híbridos. Esta etapa busca consolidar la parte computacional y teórica del proyecto mediante simulaciones robustas.

\textbf{Metas:}
\begin{itemize}
  \item Desarrollar e implementar algoritmos de ML aplicados a los modelos de colas.
  \item Analizar y validar el comportamiento de los modelos híbridos mediante pruebas computacionales.
\end{itemize}

\textbf{Entregables:}
\begin{itemize}
  \item Repositorio de código con políticas de ML implementadas.
  \item Informe técnico de análisis comparativo de políticas.
\end{itemize}

\textbf{Actividades:}
\begin{itemize}
  \item Diseño de entorno de simulación para aprendizaje por refuerzo.
  \item Implementación de políticas con TensorFlow/Keras.
  \item Ejecución de simulaciones comparativas en escenarios variables.
  \item Análisis de resultados y validación con métricas de desempeño.
\end{itemize}

\subsection*{Etapa 3 — Año 2027}
\textbf{Duración:} 12 meses\\
\textbf{Descripción:} La etapa final se enfocará en la validación empírica de los modelos desarrollados y la preparación de publicaciones científicas. Se realizarán ajustes al modelo con base en los resultados obtenidos y se elaborarán herramientas de visualización de resultados. Además, se integrará un informe final con recomendaciones, implicaciones teóricas y aplicadas, así como posibles rutas para aplicaciones futuras en sectores estratégicos.

\textbf{Metas:}
\begin{itemize}
  \item Validar los modelos en entornos simulados y ajustar parámetros finales.
  \item Difundir los resultados a través de publicaciones, presentaciones y herramientas digitales.
\end{itemize}

\textbf{Entregables:}
\begin{itemize}
  \item Artículo científico para revista indexada.
  \item Informe final y repositorio con visualización de resultados.
\end{itemize}

\textbf{Actividades:}
\begin{itemize}
  \item Ajuste final de hiperparámetros y simulaciones.
  \item Redacción y envío de artículo a revista académica.
  \item Generación de panel de visualización de políticas y desempeño.
  \item Entrega del informe final y cierre del proyecto.
\end{itemize}


\section{Plan de Trabajo por Etapas}

% =======================

\section{Semblanza}

\section{Semblanza}
Carlos Martínez Rodríguez es profesor-investigador en la UACM en el área de matemáticas aplicadas. Se especializa en teoría de colas, procesos estocásticos y ciencia de datos. Ha publicado en arXiv, ResearchGate y Academia.edu, contribuyendo en modelos de salud, urbanos y administrativos.

Carlos Martínez Rodríguez es profesor-investigador en la Universidad Autónoma de la Ciudad de México (UACM), adscrito al área de matemáticas aplicadas. Su labor académica se centra en la modelación estocástica, teoría de colas y procesos regenerativos, con aplicaciones interdisciplinarias en ciencia de datos, sistemas de atención, y procesos sociales.

Es autor de diversas publicaciones en plataformas académicas internacionales como \textit{arXiv}, donde ha difundido contribuciones sobre cadenas de Markov, modelos para el análisis de sistemas de salud y procesos de renovación. En \textit{ResearchGate}, mantiene un portafolio activo de investigación aplicada, incluyendo manuscritos de revisión, herramientas teóricas y aportaciones metodológicas. Además, en \textit{Academia.edu} comparte materiales docentes y avances de proyectos que buscan vincular la teoría con problemáticas actuales.

Su producción académica ha influido en el desarrollo de modelos matemáticos con impacto en procesos urbanos, epidemiológicos y administrativos. Como docente, destaca por su compromiso en la formación de jóvenes investigadores e integración de herramientas computacionales para la enseñanza de la probabilidad y la estadística.

Carlos colabora activamente en proyectos de ciencia básica y de frontera, y actualmente impulsa una línea de investigación sobre la incorporación de aprendizaje automático en sistemas estocásticos de servicio.

\section{Publicaciones Destacadas}
\section{Publicaciones Destacadas}
\begin{itemize}
  \item D. Campos, C.A. Martínez, A. Contreras-Cristán y F. O'Reilly (2010). \textit{Inferences for mixtures of distributions for centrally censored data with partial identification}. Communications in Statistics -- Theory and Methods, 39(12), 2241--2263.
  \item Cuellar, P., Castañeda-Ortiz, E.J., Rosales-Zarza, C., Martínez-Rodríguez, C.E., Canela-Pérez, I., Rodríguez, M.A., Valdés, J., \& Azuara-Liceaga, E. (2024). \textit{Genome-Wide Classification of Myb Domain-Containing Protein Families in Entamoeba invadens}.
\end{itemize}


\begin{itemize}
  \item D. Campos et al. (2010). \textit{Inferences for mixtures of distributions for centrally censored data...}
  \item Cuellar et al. (2024). \textit{Genome-Wide Classification of Myb Domain-Containing Protein Families...}
\end{itemize}


\section{Desglose Financiero Estimado por Etapa}

\section{Desglose Financiero Estimado por Etapa}

\section{Desglose Financiero Estimado por Etapa}

\section{Desglose Financiero Estimado por Etapa}

\section{Desglose Financiero Estimado por Etapa}

\subsection*{Etapa 1 — Año 2025}
\begin{itemize}
  \item \textbf{Adquisición de equipo de cómputo (iMac con máxima configuración)}: \$90,000 MXN
  \item \textbf{Estación de trabajo científica de alto rendimiento (procesadores múltiples, GPU, 64 GB RAM)}: \$120,000 MXN
  \item \textbf{iPad de gran capacidad (pantalla de 12.9 pulgadas)}: \$45,000 MXN
  \item \textbf{Monitor auxiliar de alta resolución}: \$25,000 MXN
  \item \textbf{Trackpad avanzado}: \$8,000 MXN
  \item \textbf{Bibliografía especializada (libros, licencias digitales)}: \$30,000 MXN
  \item \textbf{Apoyo a asistente de investigación (3 meses)}: \$30,000 MXN
  \item \textbf{Otros (papelería, servicios)}: \$10,000 MXN
  \item \textbf{Total estimado}: \textbf{\$358,000 MXN}
\end{itemize}

\subsection*{Etapa 2 — Año 2026}
\begin{itemize}
  \item \textbf{Participación en congreso nacional/internacional (viaje + inscripción)}: \$50,000 MXN
  \item \textbf{Apoyo a asistente de investigación (6 meses)}: \$60,000 MXN
  \item \textbf{Software especializado / licencias (TensorFlow, MatLab, etc.)}: \$30,000 MXN
  \item \textbf{Otros (hospedaje, materiales)}: \$10,000 MXN
  \item \textbf{Total estimado}: \textbf{\$150,000 MXN}
\end{itemize}

\subsection*{Etapa 3 — Año 2027}
\begin{itemize}
  \item \textbf{Publicación en revista indexada (open access)}: \$40,000 MXN
  \item \textbf{Desarrollo de herramienta web interactiva}: \$40,000 MXN
  \item \textbf{Apoyo a asistente de investigación (3 meses)}: \$30,000 MXN
  \item \textbf{Otros (traducción, edición, difusión)}: \$10,000 MXN
  \item \textbf{Total estimado}: \textbf{\$120,000 MXN}
\end{itemize}

\noindent\textbf{Monto total estimado del proyecto:} \textbf{\$628,000 MXN}


\subsection*{Etapa 1 — Año 2025}
\begin{itemize}
  \item \textbf{Adquisición de equipo de cómputo (iMac con máxima configuración)}: \$90,000 MXN
  \item \textbf{Adquisición de estación de trabajo científica de alto rendimiento (procesadores múltiples, GPU, 64 GB RAM)}: \$120,000 MXN
  \item \textbf{Adquisición de iPad de gran capacidad (pantalla de 12.9 pulgadas, ideal para lectura científica y anotaciones)}: \$45,000 MXN
  \item \textbf{Monitor auxiliar de alta resolución para soporte en pantallas extendidas}: \$25,000 MXN
  \item \textbf{Trackpad avanzado compatible con estación de trabajo y iMac}: \$8,000 MXN
  \item \textbf{Compra de bibliografía especializada (libros, licencias digitales)}: \$30,000 MXN
  \item \textbf{Apoyo a asistente de investigación (3 meses)}: \$30,000 MXN
  \item \textbf{Otros (papelería, servicios)}: \$10,000 MXN
  \item \textbf{Total estimado}: \textbf{\$358,000 MXN}
\end{itemize}

\subsection*{Etapa 2 — Año 2026}
\begin{itemize}
  \item \textbf{Participación en congreso nacional/internacional (viaje + inscripción)}: \$50,000 MXN
  \item \textbf{Apoyo a asistente de investigación (6 meses)}: \$60,000 MXN
  \item \textbf{Software especializado / licencias (TensorFlow, MatLab, etc.)}: \$30,000 MXN
  \item \textbf{Otros (hospedaje, materiales)}: \$10,000 MXN
  \item \textbf{Total estimado}: \textbf{\$150,000 MXN}
\end{itemize}

\subsection*{Etapa 3 — Año 2027}
\begin{itemize}
  \item \textbf{Costo de publicación en revista indexada (open access)}: \$40,000 MXN
  \item \textbf{Desarrollo y despliegue de herramienta web interactiva}: \$40,000 MXN
  \item \textbf{Apoyo a asistente de investigación (3 meses)}: \$30,000 MXN
  \item \textbf{Otros (traducción, edición, difusión)}: \$10,000 MXN
  \item \textbf{Total estimado}: \textbf{\$120,000 MXN}
\end{itemize}

\textbf{Monto total estimado del proyecto:} \textbf{\$628,000 MXN}

\subsection*{Etapa 1 — Año 2025}
\begin{itemize}
  \item \textbf{Adquisición de equipo de cómputo (iMac con máxima configuración)}: \$90,000 MXN
  \item \textbf{Adquisición de estación de trabajo científica de alto rendimiento (procesadores múltiples, GPU, 64 GB RAM)}: \$120,000 MXN
  \item \textbf{Adquisición de iPad de gran capacidad (pantalla de 12.9 pulgadas, ideal para lectura científica y anotaciones)}: \$45,000 MXN
  \item \textbf{Monitor auxiliar de alta resolución para soporte en pantallas extendidas}: \$25,000 MXN
  \item \textbf{Trackpad avanzado compatible con estación de trabajo y iMac}: \$8,000 MXN
  \item \textbf{Compra de bibliografía especializada (libros, licencias digitales)}: \$30,000 MXN
  \item \textbf{Apoyo a asistente de investigación (3 meses)}: \$30,000 MXN
  \item \textbf{Otros (papelería, servicios)}: \$10,000 MXN
  \item \textbf{Total estimado}: \textbf{\$358,000 MXN}
\end{itemize}

\subsection*{Etapa 2 — Año 2026}
\begin{itemize}
  \item \textbf{Participación en congreso nacional/internacional (viaje + inscripción)}: \$50,000 MXN
  \item \textbf{Apoyo a asistente de investigación (6 meses)}: \$60,000 MXN
  \item \textbf{Software especializado / licencias (TensorFlow, MatLab, etc.)}: \$30,000 MXN
  \item \textbf{Otros (hospedaje, materiales)}: \$10,000 MXN
  \item \textbf{Total estimado}: \textbf{\$150,000 MXN}
\end{itemize}

\subsection*{Etapa 3 — Año 2027}
\begin{itemize}
  \item \textbf{Costo de publicación en revista indexada (open access)}: \$40,000 MXN
  \item \textbf{Desarrollo y despliegue de herramienta web interactiva}: \$40,000 MXN
  \item \textbf{Apoyo a asistente de investigación (3 meses)}: \$30,000 MXN
  \item \textbf{Otros (traducción, edición, difusión)}: \$10,000 MXN
  \item \textbf{Total estimado}: \textbf{\$120,000 MXN}
\end{itemize}

\textbf{Monto total estimado del proyecto:} \textbf{\$628,000 MXN}



\subsection*{Etapa 1 — Año 2025}
\begin{itemize}
  \item \textbf{Adquisición de equipo de cómputo (iMac con máxima configuración)}: \$90,000 MXN
  \item \textbf{Adquisición de estación de trabajo científica de alto rendimiento (procesadores múltiples, GPU, 64 GB RAM)}: \$120,000 MXN
  \item \textbf{Adquisición de iPad de gran capacidad (pantalla de 12.9 pulgadas, ideal para lectura científica y anotaciones)}: \$45,000 MXN
  \item \textbf{Monitor auxiliar de alta resolución para soporte en pantallas extendidas}: \$25,000 MXN
  \item \textbf{Trackpad avanzado compatible con estación de trabajo y iMac}: \$8,000 MXN
  \item \textbf{Compra de bibliografía especializada (libros, licencias digitales)}: \$30,000 MXN
  \item \textbf{Apoyo a asistente de investigación (3 meses)}: \$30,000 MXN
  \item \textbf{Otros (papelería, servicios)}: \$10,000 MXN
  \item \textbf{Total estimado}: \textbf{\$358,000 MXN}
\end{itemize}

\subsection*{Etapa 2 — Año 2026}
\begin{itemize}
  \item \textbf{Participación en congreso nacional/internacional (viaje + inscripción)}: \$50,000 MXN
  \item \textbf{Apoyo a asistente de investigación (6 meses)}: \$60,000 MXN
  \item \textbf{Software especializado / licencias (TensorFlow, MatLab, etc.)}: \$30,000 MXN
  \item \textbf{Otros (hospedaje, materiales)}: \$10,000 MXN
  \item \textbf{Total estimado}: \textbf{\$150,000 MXN}
\end{itemize}

\subsection*{Etapa 3 — Año 2027}
\begin{itemize}
  \item \textbf{Costo de publicación en revista indexada (open access)}: \$40,000 MXN
  \item \textbf{Desarrollo y despliegue de herramienta web interactiva}: \$40,000 MXN
  \item \textbf{Apoyo a asistente de investigación (3 meses)}: \$30,000 MXN
  \item \textbf{Otros (traducción, edición, difusión)}: \$10,000 MXN
  \item \textbf{Total estimado}: \textbf{\$120,000 MXN}
\end{itemize}

\textbf{Monto total estimado del proyecto:} \textbf{\$628,000 MXN}



\subsection*{Etapa 1 — Año 2025}
\begin{itemize}
  \item \textbf{Adquisición de equipo de cómputo (iMac con máxima configuración)}: \$90,000 MXN
  \item \textbf{Adquisición de estación de trabajo científica de alto rendimiento (procesadores múltiples, GPU, 64 GB RAM)}: \$120,000 MXN
  \item \textbf{Adquisición de iPad de gran capacidad (pantalla de 12.9 pulgadas, ideal para lectura científica y anotaciones)}: \$45,000 MXN
  \item \textbf{Monitor auxiliar de alta resolución para soporte en pantallas extendidas}: \$25,000 MXN
  \item \textbf{Trackpad avanzado compatible con estación de trabajo y iMac}: \$8,000 MXN
  \item \textbf{Compra de bibliografía especializada (libros, licencias digitales)}: \$30,000 MXN
  \item \textbf{Apoyo a asistente de investigación (3 meses)}: \$30,000 MXN
  \item \textbf{Otros (papelería, servicios)}: \$10,000 MXN
  \item \textbf{Total estimado}: \textbf{\$358,000 MXN}
\end{itemize}

\subsection*{Etapa 2 — Año 2026}
\begin{itemize}
  \item \textbf{Participación en congreso nacional/internacional (viaje + inscripción)}: \$50,000 MXN
  \item \textbf{Apoyo a asistente de investigación (6 meses)}: \$60,000 MXN
  \item \textbf{Software especializado / licencias (TensorFlow, MatLab, etc.)}: \$30,000 MXN
  \item \textbf{Otros (hospedaje, materiales)}: \$10,000 MXN
  \item \textbf{Total estimado}: \textbf{\$150,000 MXN}
\end{itemize}

\subsection*{Etapa 3 — Año 2027}
\begin{itemize}
  \item \textbf{Costo de publicación en revista indexada (open access)}: \$40,000 MXN
  \item \textbf{Desarrollo y despliegue de herramienta web interactiva}: \$40,000 MXN
  \item \textbf{Apoyo a asistente de investigación (3 meses)}: \$30,000 MXN
  \item \textbf{Otros (traducción, edición, difusión)}: \$10,000 MXN
  \item \textbf{Total estimado}: \textbf{\$120,000 MXN}
\end{itemize}

\textbf{Monto total estimado del proyecto:} \textbf{\$628,000 MXN}

\end{document}
