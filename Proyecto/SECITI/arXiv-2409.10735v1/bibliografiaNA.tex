\begin{thebibliography}{99}

\bibitem{Adan} Adan, I. J. B. F., van Leeuwaarden, J. S. H., and Winands, E. M. M. (2006). On the Application of Rouche’s Theorem in Queueing Theory. \textit{Operations Research Letters}, 34(3), 355-360.

\bibitem{Asmussen} Asmussen, S. (1987). \textit{Applied Probability and Queues}. John Wiley and Sons.

\bibitem{Bhat} Bhat, N. (2008). \textit{An Introduction to Queueing Theory: Modelling and Analysis in Applications}. Birkhauser.

\bibitem{BosBoon} van den Bos, L., and Boon, M. (2013). \textit{Networks of Polling Systems} (report). Eindhoven University of Technology.

\bibitem{BoonMeiWinands} Boon, M. A. A., van der Mei, R. D., and Winands, E. M. M. (2011). Applications of Polling Systems. February 24, 2011.

\bibitem{Boxma} Boxma, J. O. (1991). Analysis and Optimization of Polling Systems, Queueing, Performance and Control in ATM, pp. 173-183.

\bibitem{Boxma2} Boxma J. O., Pseudoconservation Laws in Cyclic Service Systems, Journal of Applied Probability, vol. 24, 1987, pp. 949-964.

\bibitem{CooperI} Cooper, R. B., and Murray, G. (1969). Queues Served in Cyclic Order. \textit{The Bell System Technical Journal}, 48, 675-689.

\bibitem{Daley68} Daley, D. J. (1968). The Correlation Structure of the Output Process of Some Single Server Queueing Systems. \textit{The Annals of Mathematical Statistics}, 39(3), 1007-1019.

\bibitem{Disney} Disney, R. L., Farrell, R. L., and Morais, P. R. (1973). A Characterization of $M/G/1$ Queues with Renewal Departure Processes. \textit{Management Science}, 19(11), Theory Series, 1222-1228.

\bibitem{Down} Down, D. (1998). On the Stability of Polling Models with Multiple Servers. \textit{Journal of Applied Probability}, 35(3), 925-935.

\bibitem{Stability} Fricker, C., and Jaïbi, M. R. (1998). Stability of Multi-server Polling Models. Institute National de Recherche en Informatique et en Automatique.


\bibitem{KaspiMandelbaum} Kaspi, H., and Mandelbaum, A. (1992). Regenerative Closed Queueing Networks. \textit{Stochastics: An International Journal of Probability and Stochastic Processes}, 39(4), 239-258.

\bibitem{Kleinrock} Kleinrock, L. (1975). \textit{Queueing Systems: Theory, Volume 1}. Wiley-Interscience.

\bibitem{LevySidi} Levy, H., and Sidi, M. (1990). Polling Systems: Applications, Modeling, and Optimization. \textit{IEEE Transactions on Communications}, 30(10), 1750-1760.


\bibitem{MeynTweedie2} Meyn, S. P., and Tweedie, R. L. (1993). \textit{Markov Chains and Stochastic Stability}. Springer-Verlag.

\bibitem{MeynDown} Meyn, S. P., and Down, D. (1994). Stability of Generalized Jackson Networks. \textit{The Annals of Applied Probability}.


\bibitem{Roubos} Roubos, A. (2007). \textit{Polling Systems and their Applications}. Vrije Universiteit Amsterdam.

\bibitem{Serfozo} Serfozo, R. (2009). \textit{Basics of Applied Stochastic Processes}. Springer-Verlag.

\bibitem{Sharpe} Sharpe, M. (1998). \textit{General Theory of Markov Processes}. Academic.


\bibitem{SidiLevy1} Sidi, M., and Levy, H. (1990). Customer Routing in Polling Systems. In P. King, I. Mitrani, and R. Pooley (Eds.), \textit{Proceedings Performance '90}. North-Holland, Amsterdam.

\bibitem{SidiLevy2} Sidi, M., and Levy, H. (1991). Polling Systems with Simultaneous Arrivals. \textit{IEEE Transactions on Communications}, 39(6), 823-827.

\bibitem{TakagiI} Takagi, H., and Kleinrock, L. (1986). Analysis of Polling Systems. Cambridge: MIT Press.

\bibitem{TakagiII} Takagi, H. (1988). Queueing Analysis of Polling Models. \textit{ACM Computing Surveys}, 20(1), 5-28.
\bibitem{MeiBorst} van der Mei, R. D., and Borst, S. C. (1997). Analysis of Multiple-server Polling Systems by Means of the Power-Series Algorithm. \textit{Stochastic Models}, 13(2), 339-369.


\bibitem{Semenova} Vishnevskii, V. M., and Semenova, O. V. (2006). Mathematical Methods to Study the Polling Systems. \textit{Automation and Remote Control}, 67(2), 173-220.

\bibitem{Winands} Winands, E. M. M., Adan, I. J. B. F., and van Houtum, G. J. (2006). Mean Value Analysis for Polling Systems. \textit{Queueing Systems}, 54, 35-44.
\end{thebibliography}
