\documentclass[12pt]{article}
\usepackage[utf8]{inputenc}
\usepackage[spanish]{babel}
\usepackage{amsmath, amssymb, amsfonts}
\usepackage{geometry}
\usepackage{graphicx}
\usepackage{hyperref}
\geometry{letterpaper, margin=2.5cm}

\title{Propuesta de Investigación 2025\\Integración de Aprendizaje Automático en Sistemas de Colas con Encuestas}
\author{Carlos [Tu Apellido]\\Universidad Autónoma de la Ciudad de México}
\date{Abril 2025}

\begin{document}

\maketitle

% Secciones anteriores...

\section*{Plan de Trabajo por Etapas}

\subsection*{Etapa 1 — Año 2025}
\textbf{Duración:} 6 meses\\
\textbf{Descripción:} Durante esta primera fase se establecerán las bases conceptuales y matemáticas del proyecto. Se realizará una revisión exhaustiva de la literatura en teoría de colas, procesos estocásticos, y aprendizaje por refuerzo. Posteriormente, se formularán modelos base de polling systems con estructuras dinámicas, y se definirán formalmente las variables relevantes y métricas de desempeño. Se comenzará con la implementación de simulaciones iniciales de políticas clásicas (round-robin, gated, exhaustive) como referencia para análisis comparativo.

\textbf{Metas:}
\begin{itemize}
  \item Realizar revisión bibliográfica y sistematización de modelos existentes de colas y aprendizaje automático.
  \item Formular modelos matemáticos estocásticos básicos y definir métricas clave de desempeño.
\end{itemize}

\textbf{Entregables:}
\begin{itemize}
  \item Documento técnico con marco teórico y estado del arte.
  \item Prototipo de simulación inicial de políticas tradicionales en sistemas de colas.
\end{itemize}

\textbf{Actividades:}
\begin{itemize}
  \item Revisión de al menos 50 artículos clave sobre ML y teoría de colas.
  \item Modelado en papel y validación formal de condiciones de estabilidad.
  \item Implementación de modelos clásicos con Python/SimPy.
  \item Redacción del informe técnico de avance.
\end{itemize}

\subsection*{Etapa 2 — Año 2026}
\textbf{Duración:} 12 meses\\
\textbf{Descripción:} En esta segunda fase se desarrollarán e implementarán algoritmos de aprendizaje automático, en particular aprendizaje por refuerzo, aplicados a los modelos de colas formulados. Se evaluarán distintas arquitecturas de agentes de decisión adaptativa, midiendo su desempeño contra políticas tradicionales en diversos escenarios simulados. También se realizarán análisis matemáticos sobre estabilidad, convergencia y ergodicidad de los modelos híbridos. Esta etapa busca consolidar la parte computacional y teórica del proyecto mediante simulaciones robustas.

\textbf{Metas:}
\begin{itemize}
  \item Desarrollar e implementar algoritmos de ML aplicados a los modelos de colas.
  \item Analizar y validar el comportamiento de los modelos híbridos mediante pruebas computacionales.
\end{itemize}

\textbf{Entregables:}
\begin{itemize}
  \item Repositorio de código con políticas de ML implementadas.
  \item Informe técnico de análisis comparativo de políticas.
\end{itemize}

\textbf{Actividades:}
\begin{itemize}
  \item Diseño de entorno de simulación para aprendizaje por refuerzo.
  \item Implementación de políticas con TensorFlow/Keras.
  \item Ejecución de simulaciones comparativas en escenarios variables.
  \item Análisis de resultados y validación con métricas de desempeño.
\end{itemize}

\subsection*{Etapa 3 — Año 2027}
\textbf{Duración:} 12 meses\\
\textbf{Descripción:} La etapa final se enfocará en la validación empírica de los modelos desarrollados y la preparación de publicaciones científicas. Se realizarán ajustes al modelo con base en los resultados obtenidos y se elaborarán herramientas de visualización de resultados. Además, se integrará un informe final con recomendaciones, implicaciones teóricas y aplicadas, así como posibles rutas para aplicaciones futuras en sectores estratégicos.

\textbf{Metas:}
\begin{itemize}
  \item Validar los modelos en entornos simulados y ajustar parámetros finales.
  \item Difundir los resultados a través de publicaciones, presentaciones y herramientas digitales.
\end{itemize}

\textbf{Entregables:}
\begin{itemize}
  \item Artículo científico para revista indexada.
  \item Informe final y repositorio con visualización de resultados.
\end{itemize}

\textbf{Actividades:}
\begin{itemize}
  \item Ajuste final de hiperparámetros y simulaciones.
  \item Redacción y envío de artículo a revista académica.
  \item Generación de panel de visualización de políticas y desempeño.
  \item Entrega del informe final y cierre del proyecto.
\end{itemize}

\end{document}