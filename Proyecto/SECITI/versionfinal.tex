\documentclass[12pt]{article}
\usepackage[utf8]{inputenc}
\usepackage[spanish]{babel}
\usepackage{amsmath, amssymb, amsfonts}
\usepackage{geometry}
\usepackage{graphicx}
\usepackage{hyperref}
\geometry{letterpaper, margin=2.5cm}

\title{Propuesta de Investigación 2025\\Integración de Aprendizaje Automático en Sistemas de Colas con Encuestas}
\author{Carlos [Tu Apellido]\\Universidad Autónoma de la Ciudad de México}
\date{Abril 2025}

\begin{document}

\maketitle

% Secciones anteriores omitidas para brevedad

\section*{Plan de Trabajo por Etapas}

\subsection*{Etapa 1 — Año 2025 (Duración: 6 meses)}
\textbf{Descripción:} Durante esta primera fase se establecerán las bases conceptuales y matemáticas del proyecto. Se realizará una revisión exhaustiva de la literatura en teoría de colas, procesos estocásticos, y aprendizaje por refuerzo. Posteriormente, se formularán modelos base de polling systems con estructuras dinámicas, y se definirán formalmente las variables relevantes y métricas de desempeño. Se comenzará con la implementación de simulaciones iniciales de políticas clásicas (round-robin, gated, exhaustive) como referencia para análisis comparativo.

\textbf{Metas:}
\begin{itemize}
  \item Realizar revisión bibliográfica y sistematización de modelos existentes de colas y aprendizaje automático.
  \item Formular modelos matemáticos estocásticos básicos y definir métricas clave de desempeño.
\end{itemize}

\textbf{Entregables:}
\begin{itemize}
  \item Documento técnico con marco teórico y estado del arte.
  \item Prototipo de simulación inicial de políticas tradicionales en sistemas de colas.
\end{itemize}

\textbf{Actividades:}
\begin{itemize}
  \item Revisión de al menos 50 artículos clave sobre ML y teoría de colas. (97 caracteres)
  \item Modelado en papel y validación formal de condiciones de estabilidad. (91 caracteres)
  \item Implementación de modelos clásicos con Python/SimPy. (67 caracteres)
  \item Redacción del informe técnico de avance. (49 caracteres)
\end{itemize}

\subsection*{Etapa 2 — Año 2026 (Duración: 6 meses)}
\textbf{Descripción:} Esta fase se centrará en la implementación computacional de los modelos propuestos y en el desarrollo de algoritmos de aprendizaje automático, particularmente aprendizaje por refuerzo. Se diseñarán entornos de simulación para probar políticas de servicio aprendidas dinámicamente y se compararán con políticas tradicionales. También se trabajará en la recolección de datos simulados y en la evaluación del rendimiento de los modelos mediante métricas cuantitativas.

\textbf{Metas:}
\begin{itemize}
  \item Desarrollar e implementar algoritmos de ML adaptados a sistemas de colas.
  \item Diseñar un entorno de simulación robusto para pruebas comparativas.
\end{itemize}

\textbf{Entregables:}
\begin{itemize}
  \item Código funcional de algoritmos de aprendizaje por refuerzo.
  \item Informe de simulación con análisis comparativo y visualizaciones.
\end{itemize}

\textbf{Actividades:}
\begin{itemize}
  \item Codificación de entorno de simulación en Python con ML. (61 caracteres)
  \item Integración de modelos de aprendizaje tipo Q-Learning. (59 caracteres)
  \item Generación de conjuntos de datos sintéticos controlados. (58 caracteres)
  \item Redacción del informe de resultados y visualizaciones. (56 caracteres)
\end{itemize}

\subsection*{Etapa 3 — Año 2027 (Duración: 6 meses)}
\textbf{Descripción:} La última etapa estará enfocada en la validación teórica y práctica de los resultados obtenidos. Se realizarán análisis de estabilidad, convergencia y desempeño. Se prepararán artículos científicos para su publicación en revistas de alto impacto. También se desarrollará una herramienta de visualización interactiva y se socializarán los hallazgos con la comunidad académica.

\textbf{Metas:}
\begin{itemize}
  \item Validar matemáticamente los modelos desarrollados.
  \item Publicar y difundir los resultados científicos y técnicos del proyecto.
\end{itemize}

\textbf{Entregables:}
\begin{itemize}
  \item Artículo científico sometido a revista indexada.
  \item Herramienta de visualización y reporte final del proyecto.
\end{itemize}

\textbf{Actividades:}
\begin{itemize}
  \item Evaluación formal de condiciones de estabilidad. (52 caracteres)
  \item Envío de artículo a revista científica en matemáticas aplicadas. (68 caracteres)
  \item Diseño y documentación de herramienta visual en Shiny o Dash. (64 caracteres)
  \item Presentación de resultados en coloquios y congresos. (52 caracteres)
\end{itemize}

\end{document}
